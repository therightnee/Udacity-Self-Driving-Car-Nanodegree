
% Default to the notebook output style

    


% Inherit from the specified cell style.




    
\documentclass[11pt]{article}

    
    
    \usepackage[T1]{fontenc}
    % Nicer default font (+ math font) than Computer Modern for most use cases
    \usepackage{mathpazo}

    % Basic figure setup, for now with no caption control since it's done
    % automatically by Pandoc (which extracts ![](path) syntax from Markdown).
    \usepackage{graphicx}
    % We will generate all images so they have a width \maxwidth. This means
    % that they will get their normal width if they fit onto the page, but
    % are scaled down if they would overflow the margins.
    \makeatletter
    \def\maxwidth{\ifdim\Gin@nat@width>\linewidth\linewidth
    \else\Gin@nat@width\fi}
    \makeatother
    \let\Oldincludegraphics\includegraphics
    % Set max figure width to be 80% of text width, for now hardcoded.
    \renewcommand{\includegraphics}[1]{\Oldincludegraphics[width=.8\maxwidth]{#1}}
    % Ensure that by default, figures have no caption (until we provide a
    % proper Figure object with a Caption API and a way to capture that
    % in the conversion process - todo).
    \usepackage{caption}
    \DeclareCaptionLabelFormat{nolabel}{}
    \captionsetup{labelformat=nolabel}

    \usepackage{adjustbox} % Used to constrain images to a maximum size 
    \usepackage{xcolor} % Allow colors to be defined
    \usepackage{enumerate} % Needed for markdown enumerations to work
    \usepackage{geometry} % Used to adjust the document margins
    \usepackage{amsmath} % Equations
    \usepackage{amssymb} % Equations
    \usepackage{textcomp} % defines textquotesingle
    % Hack from http://tex.stackexchange.com/a/47451/13684:
    \AtBeginDocument{%
        \def\PYZsq{\textquotesingle}% Upright quotes in Pygmentized code
    }
    \usepackage{upquote} % Upright quotes for verbatim code
    \usepackage{eurosym} % defines \euro
    \usepackage[mathletters]{ucs} % Extended unicode (utf-8) support
    \usepackage[utf8x]{inputenc} % Allow utf-8 characters in the tex document
    \usepackage{fancyvrb} % verbatim replacement that allows latex
    \usepackage{grffile} % extends the file name processing of package graphics 
                         % to support a larger range 
    % The hyperref package gives us a pdf with properly built
    % internal navigation ('pdf bookmarks' for the table of contents,
    % internal cross-reference links, web links for URLs, etc.)
    \usepackage{hyperref}
    \usepackage{longtable} % longtable support required by pandoc >1.10
    \usepackage{booktabs}  % table support for pandoc > 1.12.2
    \usepackage[inline]{enumitem} % IRkernel/repr support (it uses the enumerate* environment)
    \usepackage[normalem]{ulem} % ulem is needed to support strikethroughs (\sout)
                                % normalem makes italics be italics, not underlines
    

    
    
    % Colors for the hyperref package
    \definecolor{urlcolor}{rgb}{0,.145,.698}
    \definecolor{linkcolor}{rgb}{.71,0.21,0.01}
    \definecolor{citecolor}{rgb}{.12,.54,.11}

    % ANSI colors
    \definecolor{ansi-black}{HTML}{3E424D}
    \definecolor{ansi-black-intense}{HTML}{282C36}
    \definecolor{ansi-red}{HTML}{E75C58}
    \definecolor{ansi-red-intense}{HTML}{B22B31}
    \definecolor{ansi-green}{HTML}{00A250}
    \definecolor{ansi-green-intense}{HTML}{007427}
    \definecolor{ansi-yellow}{HTML}{DDB62B}
    \definecolor{ansi-yellow-intense}{HTML}{B27D12}
    \definecolor{ansi-blue}{HTML}{208FFB}
    \definecolor{ansi-blue-intense}{HTML}{0065CA}
    \definecolor{ansi-magenta}{HTML}{D160C4}
    \definecolor{ansi-magenta-intense}{HTML}{A03196}
    \definecolor{ansi-cyan}{HTML}{60C6C8}
    \definecolor{ansi-cyan-intense}{HTML}{258F8F}
    \definecolor{ansi-white}{HTML}{C5C1B4}
    \definecolor{ansi-white-intense}{HTML}{A1A6B2}

    % commands and environments needed by pandoc snippets
    % extracted from the output of `pandoc -s`
    \providecommand{\tightlist}{%
      \setlength{\itemsep}{0pt}\setlength{\parskip}{0pt}}
    \DefineVerbatimEnvironment{Highlighting}{Verbatim}{commandchars=\\\{\}}
    % Add ',fontsize=\small' for more characters per line
    \newenvironment{Shaded}{}{}
    \newcommand{\KeywordTok}[1]{\textcolor[rgb]{0.00,0.44,0.13}{\textbf{{#1}}}}
    \newcommand{\DataTypeTok}[1]{\textcolor[rgb]{0.56,0.13,0.00}{{#1}}}
    \newcommand{\DecValTok}[1]{\textcolor[rgb]{0.25,0.63,0.44}{{#1}}}
    \newcommand{\BaseNTok}[1]{\textcolor[rgb]{0.25,0.63,0.44}{{#1}}}
    \newcommand{\FloatTok}[1]{\textcolor[rgb]{0.25,0.63,0.44}{{#1}}}
    \newcommand{\CharTok}[1]{\textcolor[rgb]{0.25,0.44,0.63}{{#1}}}
    \newcommand{\StringTok}[1]{\textcolor[rgb]{0.25,0.44,0.63}{{#1}}}
    \newcommand{\CommentTok}[1]{\textcolor[rgb]{0.38,0.63,0.69}{\textit{{#1}}}}
    \newcommand{\OtherTok}[1]{\textcolor[rgb]{0.00,0.44,0.13}{{#1}}}
    \newcommand{\AlertTok}[1]{\textcolor[rgb]{1.00,0.00,0.00}{\textbf{{#1}}}}
    \newcommand{\FunctionTok}[1]{\textcolor[rgb]{0.02,0.16,0.49}{{#1}}}
    \newcommand{\RegionMarkerTok}[1]{{#1}}
    \newcommand{\ErrorTok}[1]{\textcolor[rgb]{1.00,0.00,0.00}{\textbf{{#1}}}}
    \newcommand{\NormalTok}[1]{{#1}}
    
    % Additional commands for more recent versions of Pandoc
    \newcommand{\ConstantTok}[1]{\textcolor[rgb]{0.53,0.00,0.00}{{#1}}}
    \newcommand{\SpecialCharTok}[1]{\textcolor[rgb]{0.25,0.44,0.63}{{#1}}}
    \newcommand{\VerbatimStringTok}[1]{\textcolor[rgb]{0.25,0.44,0.63}{{#1}}}
    \newcommand{\SpecialStringTok}[1]{\textcolor[rgb]{0.73,0.40,0.53}{{#1}}}
    \newcommand{\ImportTok}[1]{{#1}}
    \newcommand{\DocumentationTok}[1]{\textcolor[rgb]{0.73,0.13,0.13}{\textit{{#1}}}}
    \newcommand{\AnnotationTok}[1]{\textcolor[rgb]{0.38,0.63,0.69}{\textbf{\textit{{#1}}}}}
    \newcommand{\CommentVarTok}[1]{\textcolor[rgb]{0.38,0.63,0.69}{\textbf{\textit{{#1}}}}}
    \newcommand{\VariableTok}[1]{\textcolor[rgb]{0.10,0.09,0.49}{{#1}}}
    \newcommand{\ControlFlowTok}[1]{\textcolor[rgb]{0.00,0.44,0.13}{\textbf{{#1}}}}
    \newcommand{\OperatorTok}[1]{\textcolor[rgb]{0.40,0.40,0.40}{{#1}}}
    \newcommand{\BuiltInTok}[1]{{#1}}
    \newcommand{\ExtensionTok}[1]{{#1}}
    \newcommand{\PreprocessorTok}[1]{\textcolor[rgb]{0.74,0.48,0.00}{{#1}}}
    \newcommand{\AttributeTok}[1]{\textcolor[rgb]{0.49,0.56,0.16}{{#1}}}
    \newcommand{\InformationTok}[1]{\textcolor[rgb]{0.38,0.63,0.69}{\textbf{\textit{{#1}}}}}
    \newcommand{\WarningTok}[1]{\textcolor[rgb]{0.38,0.63,0.69}{\textbf{\textit{{#1}}}}}
    
    
    % Define a nice break command that doesn't care if a line doesn't already
    % exist.
    \def\br{\hspace*{\fill} \\* }
    % Math Jax compatability definitions
    \def\gt{>}
    \def\lt{<}
    % Document parameters
    \title{Traffic\_Sign\_Classifier}
    
    
    

    % Pygments definitions
    
\makeatletter
\def\PY@reset{\let\PY@it=\relax \let\PY@bf=\relax%
    \let\PY@ul=\relax \let\PY@tc=\relax%
    \let\PY@bc=\relax \let\PY@ff=\relax}
\def\PY@tok#1{\csname PY@tok@#1\endcsname}
\def\PY@toks#1+{\ifx\relax#1\empty\else%
    \PY@tok{#1}\expandafter\PY@toks\fi}
\def\PY@do#1{\PY@bc{\PY@tc{\PY@ul{%
    \PY@it{\PY@bf{\PY@ff{#1}}}}}}}
\def\PY#1#2{\PY@reset\PY@toks#1+\relax+\PY@do{#2}}

\expandafter\def\csname PY@tok@ss\endcsname{\def\PY@tc##1{\textcolor[rgb]{0.10,0.09,0.49}{##1}}}
\expandafter\def\csname PY@tok@kn\endcsname{\let\PY@bf=\textbf\def\PY@tc##1{\textcolor[rgb]{0.00,0.50,0.00}{##1}}}
\expandafter\def\csname PY@tok@nt\endcsname{\let\PY@bf=\textbf\def\PY@tc##1{\textcolor[rgb]{0.00,0.50,0.00}{##1}}}
\expandafter\def\csname PY@tok@sh\endcsname{\def\PY@tc##1{\textcolor[rgb]{0.73,0.13,0.13}{##1}}}
\expandafter\def\csname PY@tok@si\endcsname{\let\PY@bf=\textbf\def\PY@tc##1{\textcolor[rgb]{0.73,0.40,0.53}{##1}}}
\expandafter\def\csname PY@tok@s1\endcsname{\def\PY@tc##1{\textcolor[rgb]{0.73,0.13,0.13}{##1}}}
\expandafter\def\csname PY@tok@sb\endcsname{\def\PY@tc##1{\textcolor[rgb]{0.73,0.13,0.13}{##1}}}
\expandafter\def\csname PY@tok@w\endcsname{\def\PY@tc##1{\textcolor[rgb]{0.73,0.73,0.73}{##1}}}
\expandafter\def\csname PY@tok@no\endcsname{\def\PY@tc##1{\textcolor[rgb]{0.53,0.00,0.00}{##1}}}
\expandafter\def\csname PY@tok@kt\endcsname{\def\PY@tc##1{\textcolor[rgb]{0.69,0.00,0.25}{##1}}}
\expandafter\def\csname PY@tok@s\endcsname{\def\PY@tc##1{\textcolor[rgb]{0.73,0.13,0.13}{##1}}}
\expandafter\def\csname PY@tok@nd\endcsname{\def\PY@tc##1{\textcolor[rgb]{0.67,0.13,1.00}{##1}}}
\expandafter\def\csname PY@tok@cp\endcsname{\def\PY@tc##1{\textcolor[rgb]{0.74,0.48,0.00}{##1}}}
\expandafter\def\csname PY@tok@sd\endcsname{\let\PY@it=\textit\def\PY@tc##1{\textcolor[rgb]{0.73,0.13,0.13}{##1}}}
\expandafter\def\csname PY@tok@vi\endcsname{\def\PY@tc##1{\textcolor[rgb]{0.10,0.09,0.49}{##1}}}
\expandafter\def\csname PY@tok@dl\endcsname{\def\PY@tc##1{\textcolor[rgb]{0.73,0.13,0.13}{##1}}}
\expandafter\def\csname PY@tok@nc\endcsname{\let\PY@bf=\textbf\def\PY@tc##1{\textcolor[rgb]{0.00,0.00,1.00}{##1}}}
\expandafter\def\csname PY@tok@sx\endcsname{\def\PY@tc##1{\textcolor[rgb]{0.00,0.50,0.00}{##1}}}
\expandafter\def\csname PY@tok@sa\endcsname{\def\PY@tc##1{\textcolor[rgb]{0.73,0.13,0.13}{##1}}}
\expandafter\def\csname PY@tok@gd\endcsname{\def\PY@tc##1{\textcolor[rgb]{0.63,0.00,0.00}{##1}}}
\expandafter\def\csname PY@tok@mf\endcsname{\def\PY@tc##1{\textcolor[rgb]{0.40,0.40,0.40}{##1}}}
\expandafter\def\csname PY@tok@nv\endcsname{\def\PY@tc##1{\textcolor[rgb]{0.10,0.09,0.49}{##1}}}
\expandafter\def\csname PY@tok@il\endcsname{\def\PY@tc##1{\textcolor[rgb]{0.40,0.40,0.40}{##1}}}
\expandafter\def\csname PY@tok@na\endcsname{\def\PY@tc##1{\textcolor[rgb]{0.49,0.56,0.16}{##1}}}
\expandafter\def\csname PY@tok@cm\endcsname{\let\PY@it=\textit\def\PY@tc##1{\textcolor[rgb]{0.25,0.50,0.50}{##1}}}
\expandafter\def\csname PY@tok@vm\endcsname{\def\PY@tc##1{\textcolor[rgb]{0.10,0.09,0.49}{##1}}}
\expandafter\def\csname PY@tok@fm\endcsname{\def\PY@tc##1{\textcolor[rgb]{0.00,0.00,1.00}{##1}}}
\expandafter\def\csname PY@tok@mb\endcsname{\def\PY@tc##1{\textcolor[rgb]{0.40,0.40,0.40}{##1}}}
\expandafter\def\csname PY@tok@mh\endcsname{\def\PY@tc##1{\textcolor[rgb]{0.40,0.40,0.40}{##1}}}
\expandafter\def\csname PY@tok@cs\endcsname{\let\PY@it=\textit\def\PY@tc##1{\textcolor[rgb]{0.25,0.50,0.50}{##1}}}
\expandafter\def\csname PY@tok@nf\endcsname{\def\PY@tc##1{\textcolor[rgb]{0.00,0.00,1.00}{##1}}}
\expandafter\def\csname PY@tok@s2\endcsname{\def\PY@tc##1{\textcolor[rgb]{0.73,0.13,0.13}{##1}}}
\expandafter\def\csname PY@tok@gs\endcsname{\let\PY@bf=\textbf}
\expandafter\def\csname PY@tok@ch\endcsname{\let\PY@it=\textit\def\PY@tc##1{\textcolor[rgb]{0.25,0.50,0.50}{##1}}}
\expandafter\def\csname PY@tok@mo\endcsname{\def\PY@tc##1{\textcolor[rgb]{0.40,0.40,0.40}{##1}}}
\expandafter\def\csname PY@tok@err\endcsname{\def\PY@bc##1{\setlength{\fboxsep}{0pt}\fcolorbox[rgb]{1.00,0.00,0.00}{1,1,1}{\strut ##1}}}
\expandafter\def\csname PY@tok@m\endcsname{\def\PY@tc##1{\textcolor[rgb]{0.40,0.40,0.40}{##1}}}
\expandafter\def\csname PY@tok@gh\endcsname{\let\PY@bf=\textbf\def\PY@tc##1{\textcolor[rgb]{0.00,0.00,0.50}{##1}}}
\expandafter\def\csname PY@tok@kp\endcsname{\def\PY@tc##1{\textcolor[rgb]{0.00,0.50,0.00}{##1}}}
\expandafter\def\csname PY@tok@gt\endcsname{\def\PY@tc##1{\textcolor[rgb]{0.00,0.27,0.87}{##1}}}
\expandafter\def\csname PY@tok@ow\endcsname{\let\PY@bf=\textbf\def\PY@tc##1{\textcolor[rgb]{0.67,0.13,1.00}{##1}}}
\expandafter\def\csname PY@tok@bp\endcsname{\def\PY@tc##1{\textcolor[rgb]{0.00,0.50,0.00}{##1}}}
\expandafter\def\csname PY@tok@k\endcsname{\let\PY@bf=\textbf\def\PY@tc##1{\textcolor[rgb]{0.00,0.50,0.00}{##1}}}
\expandafter\def\csname PY@tok@gi\endcsname{\def\PY@tc##1{\textcolor[rgb]{0.00,0.63,0.00}{##1}}}
\expandafter\def\csname PY@tok@gu\endcsname{\let\PY@bf=\textbf\def\PY@tc##1{\textcolor[rgb]{0.50,0.00,0.50}{##1}}}
\expandafter\def\csname PY@tok@nn\endcsname{\let\PY@bf=\textbf\def\PY@tc##1{\textcolor[rgb]{0.00,0.00,1.00}{##1}}}
\expandafter\def\csname PY@tok@gp\endcsname{\let\PY@bf=\textbf\def\PY@tc##1{\textcolor[rgb]{0.00,0.00,0.50}{##1}}}
\expandafter\def\csname PY@tok@cpf\endcsname{\let\PY@it=\textit\def\PY@tc##1{\textcolor[rgb]{0.25,0.50,0.50}{##1}}}
\expandafter\def\csname PY@tok@sr\endcsname{\def\PY@tc##1{\textcolor[rgb]{0.73,0.40,0.53}{##1}}}
\expandafter\def\csname PY@tok@ge\endcsname{\let\PY@it=\textit}
\expandafter\def\csname PY@tok@kc\endcsname{\let\PY@bf=\textbf\def\PY@tc##1{\textcolor[rgb]{0.00,0.50,0.00}{##1}}}
\expandafter\def\csname PY@tok@gr\endcsname{\def\PY@tc##1{\textcolor[rgb]{1.00,0.00,0.00}{##1}}}
\expandafter\def\csname PY@tok@o\endcsname{\def\PY@tc##1{\textcolor[rgb]{0.40,0.40,0.40}{##1}}}
\expandafter\def\csname PY@tok@nl\endcsname{\def\PY@tc##1{\textcolor[rgb]{0.63,0.63,0.00}{##1}}}
\expandafter\def\csname PY@tok@c\endcsname{\let\PY@it=\textit\def\PY@tc##1{\textcolor[rgb]{0.25,0.50,0.50}{##1}}}
\expandafter\def\csname PY@tok@nb\endcsname{\def\PY@tc##1{\textcolor[rgb]{0.00,0.50,0.00}{##1}}}
\expandafter\def\csname PY@tok@vc\endcsname{\def\PY@tc##1{\textcolor[rgb]{0.10,0.09,0.49}{##1}}}
\expandafter\def\csname PY@tok@kd\endcsname{\let\PY@bf=\textbf\def\PY@tc##1{\textcolor[rgb]{0.00,0.50,0.00}{##1}}}
\expandafter\def\csname PY@tok@kr\endcsname{\let\PY@bf=\textbf\def\PY@tc##1{\textcolor[rgb]{0.00,0.50,0.00}{##1}}}
\expandafter\def\csname PY@tok@c1\endcsname{\let\PY@it=\textit\def\PY@tc##1{\textcolor[rgb]{0.25,0.50,0.50}{##1}}}
\expandafter\def\csname PY@tok@sc\endcsname{\def\PY@tc##1{\textcolor[rgb]{0.73,0.13,0.13}{##1}}}
\expandafter\def\csname PY@tok@ne\endcsname{\let\PY@bf=\textbf\def\PY@tc##1{\textcolor[rgb]{0.82,0.25,0.23}{##1}}}
\expandafter\def\csname PY@tok@ni\endcsname{\let\PY@bf=\textbf\def\PY@tc##1{\textcolor[rgb]{0.60,0.60,0.60}{##1}}}
\expandafter\def\csname PY@tok@mi\endcsname{\def\PY@tc##1{\textcolor[rgb]{0.40,0.40,0.40}{##1}}}
\expandafter\def\csname PY@tok@vg\endcsname{\def\PY@tc##1{\textcolor[rgb]{0.10,0.09,0.49}{##1}}}
\expandafter\def\csname PY@tok@se\endcsname{\let\PY@bf=\textbf\def\PY@tc##1{\textcolor[rgb]{0.73,0.40,0.13}{##1}}}
\expandafter\def\csname PY@tok@go\endcsname{\def\PY@tc##1{\textcolor[rgb]{0.53,0.53,0.53}{##1}}}

\def\PYZbs{\char`\\}
\def\PYZus{\char`\_}
\def\PYZob{\char`\{}
\def\PYZcb{\char`\}}
\def\PYZca{\char`\^}
\def\PYZam{\char`\&}
\def\PYZlt{\char`\<}
\def\PYZgt{\char`\>}
\def\PYZsh{\char`\#}
\def\PYZpc{\char`\%}
\def\PYZdl{\char`\$}
\def\PYZhy{\char`\-}
\def\PYZsq{\char`\'}
\def\PYZdq{\char`\"}
\def\PYZti{\char`\~}
% for compatibility with earlier versions
\def\PYZat{@}
\def\PYZlb{[}
\def\PYZrb{]}
\makeatother


    % Exact colors from NB
    \definecolor{incolor}{rgb}{0.0, 0.0, 0.5}
    \definecolor{outcolor}{rgb}{0.545, 0.0, 0.0}



    
    % Prevent overflowing lines due to hard-to-break entities
    \sloppy 
    % Setup hyperref package
    \hypersetup{
      breaklinks=true,  % so long urls are correctly broken across lines
      colorlinks=true,
      urlcolor=urlcolor,
      linkcolor=linkcolor,
      citecolor=citecolor,
      }
    % Slightly bigger margins than the latex defaults
    
    \geometry{verbose,tmargin=1in,bmargin=1in,lmargin=1in,rmargin=1in}
    
    

    \begin{document}
    
    
    \maketitle
    
    

    
    \hypertarget{self-driving-car-engineer-nanodegree}{%
\section{Self-Driving Car Engineer
Nanodegree}\label{self-driving-car-engineer-nanodegree}}

\hypertarget{deep-learning}{%
\subsection{Deep Learning}\label{deep-learning}}

\hypertarget{project-build-a-traffic-sign-recognition-classifier}{%
\subsection{Project: Build a Traffic Sign Recognition
Classifier}\label{project-build-a-traffic-sign-recognition-classifier}}

In this notebook, a template is provided for you to implement your
functionality in stages, which is required to successfully complete this
project. If additional code is required that cannot be included in the
notebook, be sure that the Python code is successfully imported and
included in your submission if necessary.

\begin{quote}
\textbf{Note}: Once you have completed all of the code implementations,
you need to finalize your work by exporting the iPython Notebook as an
HTML document. Before exporting the notebook to html, all of the code
cells need to have been run so that reviewers can see the final
implementation and output. You can then export the notebook by using the
menu above and navigating to \n``,''\textbf{File -\textgreater{}
Download as -\textgreater{} HTML (.html)}. Include the finished document
along with this notebook as your submission.
\end{quote}

In addition to implementing code, there is a writeup to complete. The
writeup should be completed in a separate file, which can be either a
markdown file or a pdf document. There is a
\href{https://github.com/udacity/CarND-Traffic-Sign-Classifier-Project/blob/master/writeup_template.md}{write
up template} that can be used to guide the writing process. Completing
the code template and writeup template will cover all of the
\href{https://review.udacity.com/\#!/rubrics/481/view}{rubric points}
for this project.

The \href{https://review.udacity.com/\#!/rubrics/481/view}{rubric}
contains ``Stand Out Suggestions'' for enhancing the project beyond the
minimum requirements. The stand out suggestions are optional. If you
decide to pursue the ``stand out suggestions'', you can include the code
in this Ipython notebook and also discuss the results in the writeup
file.

\begin{quote}
\textbf{Note:} Code and Markdown cells can be executed using the
\textbf{Shift + Enter} keyboard shortcut. In addition, Markdown cells
can be edited by typically double-clicking the cell to enter edit mode.
\end{quote}

    \begin{center}\rule{0.5\linewidth}{\linethickness}\end{center}

\hypertarget{step-0-load-the-data}{%
\subsection{Step 0: Load The Data}\label{step-0-load-the-data}}

    \begin{Verbatim}[commandchars=\\\{\}]
{\color{incolor}In [{\color{incolor}1}]:} \PY{c+c1}{\PYZsh{} Load pickled data}
        \PY{k+kn}{import} \PY{n+nn}{pickle}
        
        \PY{n}{training\PYZus{}file} \PY{o}{=} \PY{l+s+s1}{\PYZsq{}}\PY{l+s+s1}{Traffic\PYZus{}Sign\PYZus{}Data/train.p}\PY{l+s+s1}{\PYZsq{}}
        \PY{n}{validation\PYZus{}file}\PY{o}{=} \PY{l+s+s1}{\PYZsq{}}\PY{l+s+s1}{Traffic\PYZus{}Sign\PYZus{}Data/valid.p}\PY{l+s+s1}{\PYZsq{}}
        \PY{n}{testing\PYZus{}file} \PY{o}{=} \PY{l+s+s1}{\PYZsq{}}\PY{l+s+s1}{Traffic\PYZus{}Sign\PYZus{}Data/test.p}\PY{l+s+s1}{\PYZsq{}}
        
        \PY{k}{with} \PY{n+nb}{open}\PY{p}{(}\PY{n}{training\PYZus{}file}\PY{p}{,} \PY{n}{mode}\PY{o}{=}\PY{l+s+s1}{\PYZsq{}}\PY{l+s+s1}{rb}\PY{l+s+s1}{\PYZsq{}}\PY{p}{)} \PY{k}{as} \PY{n}{f}\PY{p}{:}
            \PY{n}{train} \PY{o}{=} \PY{n}{pickle}\PY{o}{.}\PY{n}{load}\PY{p}{(}\PY{n}{f}\PY{p}{)}
        \PY{k}{with} \PY{n+nb}{open}\PY{p}{(}\PY{n}{validation\PYZus{}file}\PY{p}{,} \PY{n}{mode}\PY{o}{=}\PY{l+s+s1}{\PYZsq{}}\PY{l+s+s1}{rb}\PY{l+s+s1}{\PYZsq{}}\PY{p}{)} \PY{k}{as} \PY{n}{f}\PY{p}{:}
            \PY{n}{valid} \PY{o}{=} \PY{n}{pickle}\PY{o}{.}\PY{n}{load}\PY{p}{(}\PY{n}{f}\PY{p}{)}
        \PY{k}{with} \PY{n+nb}{open}\PY{p}{(}\PY{n}{testing\PYZus{}file}\PY{p}{,} \PY{n}{mode}\PY{o}{=}\PY{l+s+s1}{\PYZsq{}}\PY{l+s+s1}{rb}\PY{l+s+s1}{\PYZsq{}}\PY{p}{)} \PY{k}{as} \PY{n}{f}\PY{p}{:}
            \PY{n}{test} \PY{o}{=} \PY{n}{pickle}\PY{o}{.}\PY{n}{load}\PY{p}{(}\PY{n}{f}\PY{p}{)}
            
        \PY{n}{X\PYZus{}train}\PY{p}{,} \PY{n}{y\PYZus{}train} \PY{o}{=} \PY{n}{train}\PY{p}{[}\PY{l+s+s1}{\PYZsq{}}\PY{l+s+s1}{features}\PY{l+s+s1}{\PYZsq{}}\PY{p}{]}\PY{p}{,} \PY{n}{train}\PY{p}{[}\PY{l+s+s1}{\PYZsq{}}\PY{l+s+s1}{labels}\PY{l+s+s1}{\PYZsq{}}\PY{p}{]}
        \PY{n}{X\PYZus{}valid}\PY{p}{,} \PY{n}{y\PYZus{}valid} \PY{o}{=} \PY{n}{valid}\PY{p}{[}\PY{l+s+s1}{\PYZsq{}}\PY{l+s+s1}{features}\PY{l+s+s1}{\PYZsq{}}\PY{p}{]}\PY{p}{,} \PY{n}{valid}\PY{p}{[}\PY{l+s+s1}{\PYZsq{}}\PY{l+s+s1}{labels}\PY{l+s+s1}{\PYZsq{}}\PY{p}{]}
        \PY{n}{X\PYZus{}test}\PY{p}{,} \PY{n}{y\PYZus{}test} \PY{o}{=} \PY{n}{test}\PY{p}{[}\PY{l+s+s1}{\PYZsq{}}\PY{l+s+s1}{features}\PY{l+s+s1}{\PYZsq{}}\PY{p}{]}\PY{p}{,} \PY{n}{test}\PY{p}{[}\PY{l+s+s1}{\PYZsq{}}\PY{l+s+s1}{labels}\PY{l+s+s1}{\PYZsq{}}\PY{p}{]}
\end{Verbatim}


    \begin{center}\rule{0.5\linewidth}{\linethickness}\end{center}

\hypertarget{step-1-dataset-summary-exploration}{%
\subsection{Step 1: Dataset Summary \&
Exploration}\label{step-1-dataset-summary-exploration}}

The pickled data is a dictionary with 4 key/value pairs:

\begin{itemize}
\tightlist
\item
  \texttt{\textquotesingle{}features\textquotesingle{}} is a 4D array
  containing raw pixel data of the traffic sign images, (num examples,
  width, height, channels).
\item
  \texttt{\textquotesingle{}labels\textquotesingle{}} is a 1D array
  containing the label/class id of the traffic sign. The file
  \texttt{signnames.csv} contains id -\textgreater{} name mappings for
  each id.
\item
  \texttt{\textquotesingle{}sizes\textquotesingle{}} is a list
  containing tuples, (width, height) representing the original width and
  height the image.
\item
  \texttt{\textquotesingle{}coords\textquotesingle{}} is a list
  containing tuples, (x1, y1, x2, y2) representing coordinates of a
  bounding box around the sign in the image. \textbf{THESE COORDINATES
  ASSUME THE ORIGINAL IMAGE. THE PICKLED DATA CONTAINS RESIZED VERSIONS
  (32 by 32) OF THESE IMAGES}
\end{itemize}

Complete the basic data summary below. Use python, numpy and/or pandas
methods to calculate the data summary rather than hard coding the
results. For example, the
\href{http://pandas.pydata.org/pandas-docs/stable/generated/pandas.DataFrame.shape.html}{pandas
shape method} might be useful for calculating some of the summary
results.

    \hypertarget{provide-a-basic-summary-of-the-data-set-using-python-numpy-andor-pandas}{%
\subsubsection{Provide a Basic Summary of the Data Set Using Python,
Numpy and/or
Pandas}\label{provide-a-basic-summary-of-the-data-set-using-python-numpy-andor-pandas}}

    \begin{Verbatim}[commandchars=\\\{\}]
{\color{incolor}In [{\color{incolor}2}]:} \PY{c+c1}{\PYZsh{}\PYZsh{}\PYZsh{} Replace each question mark with the appropriate value. }
        \PY{c+c1}{\PYZsh{}\PYZsh{}\PYZsh{} Use python, pandas or numpy methods rather than hard coding the results}
        \PY{k+kn}{import} \PY{n+nn}{numpy} \PY{k}{as} \PY{n+nn}{np}
        
        \PY{c+c1}{\PYZsh{} TODO: Number of training examples}
        \PY{n}{n\PYZus{}train} \PY{o}{=} \PY{n+nb}{len}\PY{p}{(}\PY{n}{y\PYZus{}train}\PY{p}{)}
        
        \PY{c+c1}{\PYZsh{} TODO: Number of validation examples}
        \PY{n}{n\PYZus{}validation} \PY{o}{=} \PY{n+nb}{len}\PY{p}{(}\PY{n}{y\PYZus{}valid}\PY{p}{)}
        
        \PY{c+c1}{\PYZsh{} TODO: Number of testing examples.}
        \PY{n}{n\PYZus{}test} \PY{o}{=} \PY{n+nb}{len}\PY{p}{(}\PY{n}{y\PYZus{}test}\PY{p}{)}
        
        \PY{c+c1}{\PYZsh{} TODO: What\PYZsq{}s the shape of an traffic sign image?}
        \PY{n}{image\PYZus{}shape} \PY{o}{=} \PY{n}{test}\PY{p}{[}\PY{l+s+s1}{\PYZsq{}}\PY{l+s+s1}{sizes}\PY{l+s+s1}{\PYZsq{}}\PY{p}{]}\PY{p}{[}\PY{l+m+mi}{1}\PY{p}{]}
        
        \PY{c+c1}{\PYZsh{} TODO: How many unique classes/labels there are in the dataset.}
        \PY{n}{n\PYZus{}classes} \PY{o}{=} \PY{n+nb}{len}\PY{p}{(}\PY{n+nb}{set}\PY{p}{(}\PY{n}{test}\PY{p}{[}\PY{l+s+s1}{\PYZsq{}}\PY{l+s+s1}{labels}\PY{l+s+s1}{\PYZsq{}}\PY{p}{]}\PY{p}{)}\PY{p}{)}
        
        \PY{n+nb}{print}\PY{p}{(}\PY{l+s+s2}{\PYZdq{}}\PY{l+s+s2}{Number of training examples =}\PY{l+s+s2}{\PYZdq{}}\PY{p}{,} \PY{n}{n\PYZus{}train}\PY{p}{)}
        \PY{n+nb}{print}\PY{p}{(}\PY{l+s+s2}{\PYZdq{}}\PY{l+s+s2}{Number of testing examples =}\PY{l+s+s2}{\PYZdq{}}\PY{p}{,} \PY{n}{n\PYZus{}test}\PY{p}{)}
        \PY{n+nb}{print}\PY{p}{(}\PY{l+s+s2}{\PYZdq{}}\PY{l+s+s2}{Image data shape =}\PY{l+s+s2}{\PYZdq{}}\PY{p}{,} \PY{n}{image\PYZus{}shape}\PY{p}{)}
        \PY{n+nb}{print}\PY{p}{(}\PY{l+s+s2}{\PYZdq{}}\PY{l+s+s2}{Number of classes =}\PY{l+s+s2}{\PYZdq{}}\PY{p}{,} \PY{n}{n\PYZus{}classes}\PY{p}{)}
\end{Verbatim}


    \begin{Verbatim}[commandchars=\\\{\}]
Number of training examples = 34799
Number of testing examples = 12630
Image data shape = [42 45]
Number of classes = 43

    \end{Verbatim}

    \hypertarget{include-an-exploratory-visualization-of-the-dataset}{%
\subsubsection{Include an exploratory visualization of the
dataset}\label{include-an-exploratory-visualization-of-the-dataset}}

    Visualize the German Traffic Signs Dataset using the pickled file(s).
This is open ended, suggestions include: plotting traffic sign images,
plotting the count of each sign, etc.

The \href{http://matplotlib.org/}{Matplotlib}
\href{http://matplotlib.org/examples/index.html}{examples} and
\href{http://matplotlib.org/gallery.html}{gallery} pages are a great
resource for doing visualizations in Python.

\textbf{NOTE:} It's recommended you start with something simple first.
If you wish to do more, come back to it after you've completed the rest
of the sections. It can be interesting to look at the distribution of
classes in the training, validation and test set. Is the distribution
the same? Are there more examples of some classes than others?

    \begin{Verbatim}[commandchars=\\\{\}]
{\color{incolor}In [{\color{incolor}3}]:} \PY{c+c1}{\PYZsh{}\PYZsh{}\PYZsh{} Data exploration visualization code goes here.}
        \PY{c+c1}{\PYZsh{}\PYZsh{}\PYZsh{} Feel free to use as many code cells as needed.}
        \PY{k+kn}{import} \PY{n+nn}{matplotlib}\PY{n+nn}{.}\PY{n+nn}{pyplot} \PY{k}{as} \PY{n+nn}{plt}
        \PY{c+c1}{\PYZsh{} Visualizations will be shown in the notebook.}
        \PY{o}{\PYZpc{}}\PY{k}{matplotlib} inline
        
        \PY{c+c1}{\PYZsh{}visualize class frequency}
        \PY{n}{plt}\PY{o}{.}\PY{n}{hist}\PY{p}{(}\PY{n}{y\PYZus{}train}\PY{p}{,} \PY{n}{bins}\PY{o}{=}\PY{n}{np}\PY{o}{.}\PY{n}{arange}\PY{p}{(}\PY{n}{n\PYZus{}classes}\PY{p}{)}\PY{p}{)}
        \PY{n}{plt}\PY{o}{.}\PY{n}{hist}\PY{p}{(}\PY{n}{y\PYZus{}valid}\PY{p}{,} \PY{n}{bins}\PY{o}{=}\PY{n}{np}\PY{o}{.}\PY{n}{arange}\PY{p}{(}\PY{n}{n\PYZus{}classes}\PY{p}{)}\PY{p}{)}
        \PY{n}{plt}\PY{o}{.}\PY{n}{hist}\PY{p}{(}\PY{n}{y\PYZus{}test}\PY{p}{,} \PY{n}{bins}\PY{o}{=}\PY{n}{np}\PY{o}{.}\PY{n}{arange}\PY{p}{(}\PY{n}{n\PYZus{}classes}\PY{p}{)}\PY{p}{,} \PY{n}{alpha}\PY{o}{=}\PY{l+m+mf}{0.5}\PY{p}{)}
        \PY{n}{plt}\PY{o}{.}\PY{n}{show}\PY{p}{(}\PY{p}{)}
        
        \PY{c+c1}{\PYZsh{}calculate average image}
        \PY{n+nb}{print}\PY{p}{(}\PY{n}{X\PYZus{}train}\PY{o}{.}\PY{n}{shape}\PY{p}{)}
        \PY{n+nb}{print}\PY{p}{(}\PY{n}{np}\PY{o}{.}\PY{n}{max}\PY{p}{(}\PY{n}{X\PYZus{}train}\PY{p}{)}\PY{p}{,} \PY{n}{np}\PY{o}{.}\PY{n}{min}\PY{p}{(}\PY{n}{X\PYZus{}train}\PY{p}{)}\PY{p}{,} \PY{n}{X\PYZus{}train}\PY{o}{.}\PY{n}{dtype}\PY{p}{)}
        \PY{n}{mean\PYZus{}image} \PY{o}{=} \PY{n}{np}\PY{o}{.}\PY{n}{mean}\PY{p}{(}\PY{n}{X\PYZus{}train}\PY{p}{,} \PY{n}{axis}\PY{o}{=}\PY{l+m+mi}{0}\PY{p}{)}
        \PY{n+nb}{print}\PY{p}{(}\PY{n}{mean\PYZus{}image}\PY{o}{.}\PY{n}{dtype}\PY{p}{)}
        \PY{n}{plt}\PY{o}{.}\PY{n}{imshow}\PY{p}{(}\PY{n}{mean\PYZus{}image}\PY{o}{.}\PY{n}{astype}\PY{p}{(}\PY{n}{np}\PY{o}{.}\PY{n}{uint8}\PY{p}{)}\PY{p}{)}
        \PY{n}{plt}\PY{o}{.}\PY{n}{show}\PY{p}{(}\PY{p}{)}
        
        \PY{c+c1}{\PYZsh{}average class 10 }
        \PY{n+nb}{print}\PY{p}{(}\PY{l+s+s2}{\PYZdq{}}\PY{l+s+s2}{Average of Each Image}\PY{l+s+s2}{\PYZdq{}}\PY{p}{)}
        
        \PY{n}{fig} \PY{o}{=} \PY{n}{plt}\PY{o}{.}\PY{n}{figure}\PY{p}{(}\PY{n}{figsize} \PY{o}{=} \PY{p}{[}\PY{l+m+mi}{10}\PY{p}{,}\PY{l+m+mi}{10}\PY{p}{]}\PY{p}{)}
        \PY{k}{for} \PY{n}{i} \PY{o+ow}{in} \PY{n+nb}{range}\PY{p}{(}\PY{l+m+mi}{0}\PY{p}{,} \PY{n}{n\PYZus{}classes}\PY{p}{)}\PY{p}{:}
            \PY{n}{mean\PYZus{}image} \PY{o}{=} \PY{n}{np}\PY{o}{.}\PY{n}{mean}\PY{p}{(}\PY{n}{X\PYZus{}train}\PY{p}{[}\PY{n}{y\PYZus{}train} \PY{o}{==} \PY{n}{i}\PY{p}{]}\PY{p}{,} \PY{n}{axis}\PY{o}{=}\PY{l+m+mi}{0}\PY{p}{)}
            \PY{n}{plt}\PY{o}{.}\PY{n}{subplot}\PY{p}{(}\PY{l+m+mi}{7}\PY{p}{,}\PY{l+m+mi}{7}\PY{p}{,} \PY{n}{i}\PY{o}{+}\PY{l+m+mi}{1}\PY{p}{)}
            \PY{n}{plt}\PY{o}{.}\PY{n}{imshow}\PY{p}{(}\PY{n}{mean\PYZus{}image}\PY{o}{.}\PY{n}{astype}\PY{p}{(}\PY{n}{np}\PY{o}{.}\PY{n}{uint8}\PY{p}{)}\PY{p}{)}
        
        \PY{n}{plt}\PY{o}{.}\PY{n}{show}\PY{p}{(}\PY{p}{)}
        \PY{n+nb}{print}\PY{p}{(}\PY{l+s+s2}{\PYZdq{}}\PY{l+s+s2}{Loaded}\PY{l+s+s2}{\PYZdq{}}\PY{p}{)}
\end{Verbatim}


    \begin{center}
    \adjustimage{max size={0.9\linewidth}{0.9\paperheight}}{output_8_0.png}
    \end{center}
    { \hspace*{\fill} \\}
    
    \begin{Verbatim}[commandchars=\\\{\}]
(34799, 32, 32, 3)
255 0 uint8
float64

    \end{Verbatim}

    \begin{center}
    \adjustimage{max size={0.9\linewidth}{0.9\paperheight}}{output_8_2.png}
    \end{center}
    { \hspace*{\fill} \\}
    
    \begin{Verbatim}[commandchars=\\\{\}]
Average of Each Image

    \end{Verbatim}

    \begin{center}
    \adjustimage{max size={0.9\linewidth}{0.9\paperheight}}{output_8_4.png}
    \end{center}
    { \hspace*{\fill} \\}
    
    \begin{Verbatim}[commandchars=\\\{\}]
Loaded

    \end{Verbatim}

    \hypertarget{data-visualization}{%
\paragraph{Data Visualization}\label{data-visualization}}

I visualized the data in the following ways.

First I created a histogram of the classes and their frequencies in each
of the three data sets. This allowed me to see the magnitude of the
training set, validation set, and test set, as well as potential
overfitting pitfalls. For example, The test set seems to skew heavily
towards lower number classes (0-12), and seeing as those are circular
speed signs, the network may be trained to lean towards associating
circular signs with speed signs.

Second, I took the mean of all image values in the train set to get an
idea of the ``average'' traffic sign. As evidenced via the histogram it
is mostly circular, red bordered with a white infill, with a speed
marker in the center. The faint outline of a triangle is evident, which
tells me that triangular signs have a strong presence in the training
set.

Finally, I wanted to see the average image for every class. This
analysis gives insight into lighting condition variations between
classes, like between the 20 and 60 kph speed signs. All the images seem
relatively well centered, and without much distortion. While this means
the model will likely have an easier time learning from this data, it
may mean that models trained from this data may be brittle and
augumenting the set with shifted or distorted images may be helpful.

    \begin{center}\rule{0.5\linewidth}{\linethickness}\end{center}

\hypertarget{step-2-design-and-test-a-model-architecture}{%
\subsection{Step 2: Design and Test a Model
Architecture}\label{step-2-design-and-test-a-model-architecture}}

Design and implement a deep learning model that learns to recognize
traffic signs. Train and test your model on the
\href{http://benchmark.ini.rub.de/?section=gtsrb\&subsection=dataset}{German
Traffic Sign Dataset}.

The LeNet-5 implementation shown in the
\href{https://classroom.udacity.com/nanodegrees/nd013/parts/fbf77062-5703-404e-b60c-95b78b2f3f9e/modules/6df7ae49-c61c-4bb2-a23e-6527e69209ec/lessons/601ae704-1035-4287-8b11-e2c2716217ad/concepts/d4aca031-508f-4e0b-b493-e7b706120f81}{classroom}
at the end of the CNN lesson is a solid starting point. You'll have to
change the number of classes and possibly the preprocessing, but aside
from that it's plug and play!

With the LeNet-5 solution from the lecture, you should expect a
validation set accuracy of about 0.89. To meet specifications, the
validation set accuracy will need to be at least 0.93. It is possible to
get an even higher accuracy, but 0.93 is the minimum for a successful
project submission.

There are various aspects to consider when thinking about this problem:

\begin{itemize}
\tightlist
\item
  Neural network architecture (is the network over or underfitting?)
\item
  Play around preprocessing techniques (normalization, rgb to grayscale,
  etc)
\item
  Number of examples per label (some have more than others).
\item
  Generate fake data.
\end{itemize}

Here is an example of a
\href{http://yann.lecun.com/exdb/publis/pdf/sermanet-ijcnn-11.pdf}{published
baseline model on this problem}. It's not required to be familiar with
the approach used in the paper but, it's good practice to try to read
papers like these.

    \hypertarget{pre-process-the-data-set-normalization-grayscale-etc.}{%
\subsubsection{Pre-process the Data Set (normalization, grayscale,
etc.)}\label{pre-process-the-data-set-normalization-grayscale-etc.}}

    Minimally, the image data should be normalized so that the data has mean
zero and equal variance. For image data, \texttt{(pixel\ -\ 128)/\ 128}
is a quick way to approximately normalize the data and can be used in
this project.

Other pre-processing steps are optional. You can try different
techniques to see if it improves performance.

Use the code cell (or multiple code cells, if necessary) to implement
the first step of your project.

    \begin{Verbatim}[commandchars=\\\{\}]
{\color{incolor}In [{\color{incolor}4}]:} \PY{c+c1}{\PYZsh{}\PYZsh{}\PYZsh{} Preprocess the data here. It is required to norfrom sklearn.utils import shufflemalize the data. Other preprocessing steps could include }
        \PY{c+c1}{\PYZsh{}\PYZsh{}\PYZsh{} converting to grayscale, etc.}
        \PY{c+c1}{\PYZsh{}\PYZsh{}\PYZsh{} Feel free to use as many code cells as needed.}
        
        \PY{k+kn}{from} \PY{n+nn}{sklearn}\PY{n+nn}{.}\PY{n+nn}{utils} \PY{k}{import} \PY{n}{shuffle}
        
        \PY{n}{X\PYZus{}train}\PY{p}{,} \PY{n}{y\PYZus{}train} \PY{o}{=} \PY{n}{shuffle}\PY{p}{(}\PY{n}{X\PYZus{}train}\PY{p}{,} \PY{n}{y\PYZus{}train}\PY{p}{)}
        
        \PY{n}{X\PYZus{}train} \PY{o}{=} \PY{p}{(}\PY{n}{X\PYZus{}train} \PY{o}{\PYZhy{}} \PY{l+m+mf}{127.5}\PY{p}{)}\PY{o}{/}\PY{l+m+mf}{127.5}
        
        \PY{n}{X\PYZus{}valid} \PY{o}{=} \PY{p}{(}\PY{n}{X\PYZus{}valid} \PY{o}{\PYZhy{}} \PY{l+m+mf}{127.5}\PY{p}{)}\PY{o}{/}\PY{l+m+mf}{127.5}
        
        \PY{n}{X\PYZus{}test} \PY{o}{=} \PY{p}{(}\PY{n}{X\PYZus{}test} \PY{o}{\PYZhy{}} \PY{l+m+mf}{127.5}\PY{p}{)}\PY{o}{/}\PY{l+m+mf}{127.5}
\end{Verbatim}


    \hypertarget{rubric-question-1}{%
\paragraph{Rubric Question 1}\label{rubric-question-1}}

Describe how you preprocessed the image data. What techniques were
chosen and why did you choose these techniques? Consider including
images showing the output of each preprocessing technique.
Pre-processing refers to techniques such as converting to grayscale,
normalization, etc.

I decided to only normalize and shuffle the data. Looking at the mean
images for each class, the color variation stood out as a powerful
feature, so maintaining the RGB channels would be useful, so I decided
only to normalize the data.

While I ended up only executing a singular pre-process on the data, I
did consider a range of options such as Gaussian blurring, brightness
adjustment, and augmenting the data set by offsetting each image to
prevent overfitting to centered images. All could be useful, but I did
not feel there was much to be gained in this particular case.

    \hypertarget{model-architecture}{%
\subsubsection{Model Architecture}\label{model-architecture}}

    \begin{Verbatim}[commandchars=\\\{\}]
{\color{incolor}In [{\color{incolor}5}]:} \PY{c+c1}{\PYZsh{}\PYZsh{}\PYZsh{} Define your architecture here.}
        \PY{c+c1}{\PYZsh{}\PYZsh{}\PYZsh{} Feel free to use as many code cells as needed.}
        \PY{k+kn}{import} \PY{n+nn}{tensorflow} \PY{k}{as} \PY{n+nn}{tf}
        \PY{k+kn}{from} \PY{n+nn}{tensorflow}\PY{n+nn}{.}\PY{n+nn}{contrib}\PY{n+nn}{.}\PY{n+nn}{layers} \PY{k}{import} \PY{n}{flatten}
        
        \PY{n}{input\PYZus{}data} \PY{o}{=} \PY{n}{tf}\PY{o}{.}\PY{n}{placeholder}\PY{p}{(}\PY{n}{tf}\PY{o}{.}\PY{n}{float32}\PY{p}{,} \PY{p}{[}\PY{k+kc}{None}\PY{p}{,}\PY{l+m+mi}{32}\PY{p}{,}\PY{l+m+mi}{32}\PY{p}{,}\PY{l+m+mi}{3}\PY{p}{]}\PY{p}{)}
        \PY{n}{label\PYZus{}data} \PY{o}{=} \PY{n}{tf}\PY{o}{.}\PY{n}{placeholder}\PY{p}{(}\PY{n}{tf}\PY{o}{.}\PY{n}{int32}\PY{p}{,} \PY{p}{[}\PY{k+kc}{None}\PY{p}{]}\PY{p}{)}
        
        \PY{c+c1}{\PYZsh{}mean and sigma variable}
        \PY{n}{mu} \PY{o}{=} \PY{l+m+mi}{0}
        \PY{n}{sigma} \PY{o}{=} \PY{l+m+mf}{0.1}
        
        \PY{k}{def} \PY{n+nf}{conv\PYZus{}layer}\PY{p}{(}\PY{n}{input\PYZus{}data}\PY{p}{,}\PY{n}{h}\PY{p}{,}\PY{n}{w}\PY{p}{,}\PY{n}{d}\PY{p}{,}\PY{n}{fd}\PY{p}{)}\PY{p}{:}
            \PY{n}{W} \PY{o}{=} \PY{n}{tf}\PY{o}{.}\PY{n}{Variable}\PY{p}{(}\PY{n}{tf}\PY{o}{.}\PY{n}{truncated\PYZus{}normal}\PY{p}{(}\PY{n}{shape} \PY{o}{=} \PY{p}{(}\PY{n}{h}\PY{p}{,}\PY{n}{w}\PY{p}{,}\PY{n}{d}\PY{p}{,}\PY{n}{fd}\PY{p}{)}\PY{p}{,} \PY{n}{mean}\PY{o}{=}\PY{n}{mu}\PY{p}{,} \PY{n}{stddev}\PY{o}{=}\PY{n}{sigma}\PY{p}{)}\PY{p}{,} \PY{n}{trainable}\PY{o}{=}\PY{k+kc}{True}\PY{p}{)}
            \PY{n}{b} \PY{o}{=} \PY{n}{tf}\PY{o}{.}\PY{n}{Variable}\PY{p}{(}\PY{n}{tf}\PY{o}{.}\PY{n}{zeros}\PY{p}{(}\PY{n}{fd}\PY{p}{)}\PY{p}{,} \PY{n}{trainable}\PY{o}{=}\PY{k+kc}{True}\PY{p}{)}
            \PY{n}{conv} \PY{o}{=} \PY{n}{tf}\PY{o}{.}\PY{n}{nn}\PY{o}{.}\PY{n}{relu}\PY{p}{(}\PY{n}{tf}\PY{o}{.}\PY{n}{nn}\PY{o}{.}\PY{n}{conv2d}\PY{p}{(}\PY{n}{input\PYZus{}data}\PY{p}{,} \PY{n}{W}\PY{p}{,} \PY{n}{strides}\PY{o}{=}\PY{p}{[}\PY{l+m+mi}{1}\PY{p}{,} \PY{l+m+mi}{1}\PY{p}{,} \PY{l+m+mi}{1}\PY{p}{,} \PY{l+m+mi}{1}\PY{p}{]}\PY{p}{,} \PY{n}{padding}\PY{o}{=}\PY{l+s+s1}{\PYZsq{}}\PY{l+s+s1}{SAME}\PY{l+s+s1}{\PYZsq{}}\PY{p}{)} \PY{o}{+} \PY{n}{b}\PY{p}{)}
            \PY{k}{return} \PY{n}{conv}
        
        \PY{k}{def} \PY{n+nf}{mpool}\PY{p}{(}\PY{n}{input\PYZus{}data}\PY{p}{)}\PY{p}{:}
            \PY{n}{mpool} \PY{o}{=} \PY{n}{tf}\PY{o}{.}\PY{n}{nn}\PY{o}{.}\PY{n}{max\PYZus{}pool}\PY{p}{(}\PY{n}{input\PYZus{}data}\PY{p}{,} \PY{n}{ksize}\PY{o}{=}\PY{p}{[}\PY{l+m+mi}{1}\PY{p}{,} \PY{l+m+mi}{2}\PY{p}{,} \PY{l+m+mi}{2}\PY{p}{,} \PY{l+m+mi}{1}\PY{p}{]}\PY{p}{,} \PY{n}{strides}\PY{o}{=}\PY{p}{[}\PY{l+m+mi}{1}\PY{p}{,} \PY{l+m+mi}{2}\PY{p}{,} \PY{l+m+mi}{2}\PY{p}{,} \PY{l+m+mi}{1}\PY{p}{]}\PY{p}{,} \PY{n}{padding}\PY{o}{=}\PY{l+s+s1}{\PYZsq{}}\PY{l+s+s1}{VALID}\PY{l+s+s1}{\PYZsq{}}\PY{p}{)}
            \PY{k}{return} \PY{n}{mpool}
        
        \PY{c+c1}{\PYZsh{}Layer One \PYZhy{} Convolution 1}
        \PY{n}{l1c1} \PY{o}{=} \PY{n}{conv\PYZus{}layer}\PY{p}{(}\PY{n}{input\PYZus{}data}\PY{p}{,}\PY{l+m+mi}{3}\PY{p}{,}\PY{l+m+mi}{3}\PY{p}{,}\PY{l+m+mi}{3}\PY{p}{,}\PY{l+m+mi}{32}\PY{p}{)}
        \PY{n}{l1c2} \PY{o}{=} \PY{n}{conv\PYZus{}layer}\PY{p}{(}\PY{n}{l1c1}\PY{p}{,}\PY{l+m+mi}{3}\PY{p}{,}\PY{l+m+mi}{3}\PY{p}{,}\PY{l+m+mi}{32}\PY{p}{,}\PY{l+m+mi}{32}\PY{p}{)}
        \PY{n}{l1c3} \PY{o}{=} \PY{n}{conv\PYZus{}layer}\PY{p}{(}\PY{n}{l1c2}\PY{p}{,}\PY{l+m+mi}{3}\PY{p}{,}\PY{l+m+mi}{3}\PY{p}{,}\PY{l+m+mi}{32}\PY{p}{,}\PY{l+m+mi}{32}\PY{p}{)}
        
        \PY{c+c1}{\PYZsh{}Layer Two \PYZhy{} Max Pool 1}
        \PY{n}{l2\PYZus{}mp} \PY{o}{=} \PY{n}{mpool}\PY{p}{(}\PY{n}{l1c3}\PY{p}{)}
        
        \PY{c+c1}{\PYZsh{}Layer Three \PYZhy{} Convolution 2}
        \PY{n}{l3c1} \PY{o}{=} \PY{n}{conv\PYZus{}layer}\PY{p}{(}\PY{n}{l2\PYZus{}mp}\PY{p}{,}\PY{l+m+mi}{3}\PY{p}{,}\PY{l+m+mi}{3}\PY{p}{,}\PY{l+m+mi}{32}\PY{p}{,}\PY{l+m+mi}{64}\PY{p}{)}
        \PY{n}{l3c2} \PY{o}{=} \PY{n}{conv\PYZus{}layer}\PY{p}{(}\PY{n}{l3c1}\PY{p}{,}\PY{l+m+mi}{3}\PY{p}{,}\PY{l+m+mi}{3}\PY{p}{,}\PY{l+m+mi}{64}\PY{p}{,}\PY{l+m+mi}{64}\PY{p}{)}
        \PY{n}{l3c3} \PY{o}{=} \PY{n}{conv\PYZus{}layer}\PY{p}{(}\PY{n}{l3c2}\PY{p}{,}\PY{l+m+mi}{3}\PY{p}{,}\PY{l+m+mi}{3}\PY{p}{,}\PY{l+m+mi}{64}\PY{p}{,}\PY{l+m+mi}{64}\PY{p}{)}
        
        \PY{c+c1}{\PYZsh{}Layer Four \PYZhy{} Max Pool 2}
        \PY{n}{l4\PYZus{}mp} \PY{o}{=} \PY{n}{mpool}\PY{p}{(}\PY{n}{l3c3}\PY{p}{)}
        
        \PY{c+c1}{\PYZsh{}Layer Five \PYZhy{} Convolution 3}
        \PY{n}{l5c1} \PY{o}{=} \PY{n}{conv\PYZus{}layer}\PY{p}{(}\PY{n}{l4\PYZus{}mp}\PY{p}{,}\PY{l+m+mi}{3}\PY{p}{,}\PY{l+m+mi}{3}\PY{p}{,}\PY{l+m+mi}{64}\PY{p}{,}\PY{l+m+mi}{128}\PY{p}{)}
        \PY{n}{l5c2} \PY{o}{=} \PY{n}{conv\PYZus{}layer}\PY{p}{(}\PY{n}{l5c1}\PY{p}{,}\PY{l+m+mi}{3}\PY{p}{,}\PY{l+m+mi}{3}\PY{p}{,}\PY{l+m+mi}{128}\PY{p}{,}\PY{l+m+mi}{128}\PY{p}{)}
        \PY{n}{l5c3} \PY{o}{=} \PY{n}{conv\PYZus{}layer}\PY{p}{(}\PY{n}{l5c2}\PY{p}{,}\PY{l+m+mi}{3}\PY{p}{,}\PY{l+m+mi}{3}\PY{p}{,}\PY{l+m+mi}{128}\PY{p}{,}\PY{l+m+mi}{128}\PY{p}{)}
        
        \PY{c+c1}{\PYZsh{}Layer Six \PYZhy{} Avg Pool 1}
        \PY{n}{l6\PYZus{}mp} \PY{o}{=} \PY{n}{tf}\PY{o}{.}\PY{n}{nn}\PY{o}{.}\PY{n}{avg\PYZus{}pool}\PY{p}{(}\PY{n}{l5c3}\PY{p}{,} \PY{n}{ksize}\PY{o}{=}\PY{p}{[}\PY{l+m+mi}{1}\PY{p}{,} \PY{l+m+mi}{8}\PY{p}{,} \PY{l+m+mi}{8}\PY{p}{,} \PY{l+m+mi}{1}\PY{p}{]}\PY{p}{,} \PY{n}{strides}\PY{o}{=}\PY{p}{[}\PY{l+m+mi}{1}\PY{p}{,} \PY{l+m+mi}{8}\PY{p}{,} \PY{l+m+mi}{8}\PY{p}{,} \PY{l+m+mi}{1}\PY{p}{]}\PY{p}{,} \PY{n}{padding}\PY{o}{=}\PY{l+s+s1}{\PYZsq{}}\PY{l+s+s1}{VALID}\PY{l+s+s1}{\PYZsq{}}\PY{p}{)}
        
        \PY{c+c1}{\PYZsh{}Layer Seven \PYZhy{} Flatten}
        \PY{n}{l7\PYZus{}f} \PY{o}{=} \PY{n}{flatten}\PY{p}{(}\PY{n}{l6\PYZus{}mp}\PY{p}{)}
        
        \PY{c+c1}{\PYZsh{}Layer Eight \PYZhy{} Fully Connected Linear Output}
        \PY{n}{clf\PYZus{}W}  \PY{o}{=} \PY{n}{tf}\PY{o}{.}\PY{n}{Variable}\PY{p}{(}\PY{n}{tf}\PY{o}{.}\PY{n}{truncated\PYZus{}normal}\PY{p}{(}\PY{n}{shape}\PY{o}{=}\PY{p}{(}\PY{n}{l7\PYZus{}f}\PY{o}{.}\PY{n}{get\PYZus{}shape}\PY{p}{(}\PY{p}{)}\PY{o}{.}\PY{n}{as\PYZus{}list}\PY{p}{(}\PY{p}{)}\PY{p}{[}\PY{o}{\PYZhy{}}\PY{l+m+mi}{1}\PY{p}{]}\PY{p}{,} \PY{n}{n\PYZus{}classes}\PY{p}{)}\PY{p}{,} \PY{n}{mean} \PY{o}{=} \PY{n}{mu}\PY{p}{,} \PY{n}{stddev} \PY{o}{=} \PY{n}{sigma}\PY{p}{)}\PY{p}{,} \PY{n}{trainable}\PY{o}{=}\PY{k+kc}{True}\PY{p}{)}
        \PY{n}{clf\PYZus{}b}  \PY{o}{=} \PY{n}{tf}\PY{o}{.}\PY{n}{Variable}\PY{p}{(}\PY{n}{tf}\PY{o}{.}\PY{n}{zeros}\PY{p}{(}\PY{n}{n\PYZus{}classes}\PY{p}{)}\PY{p}{,} \PY{n}{trainable}\PY{o}{=}\PY{k+kc}{True}\PY{p}{)}
        \PY{n}{logits} \PY{o}{=} \PY{n}{tf}\PY{o}{.}\PY{n}{matmul}\PY{p}{(}\PY{n}{l7\PYZus{}f}\PY{p}{,} \PY{n}{clf\PYZus{}W}\PY{p}{)} \PY{o}{+} \PY{n}{clf\PYZus{}b}
            
\end{Verbatim}


    \begin{Verbatim}[commandchars=\\\{\}]
/home/alec/anaconda3/envs/dev/lib/python3.6/importlib/\_bootstrap.py:205: RuntimeWarning: compiletime version 3.5 of module 'tensorflow.python.framework.fast\_tensor\_util' does not match runtime version 3.6
  return f(*args, **kwds)

    \end{Verbatim}

    \hypertarget{description-of-the-architecture}{%
\paragraph{Description of the
Architecture}\label{description-of-the-architecture}}

Describe what your final model architecture looks like including model
type, layers, layer sizes, connectivity, etc.) Consider including a
diagram and/or table describing the final model.

The model I created is inspired by the VGG16 design, shown in the figure
below.

There are 9 layers in total.

Layer 1: 3 convlutions with RELU activation. 3x3x3 (3x3 sample size and
3 channels deep for the RGB), and I start with 16 filters. VGG begins
with 128, but I wanted a less computationally intensive model so I could
at least solve a single epoch on my CPU before moving to an AWS instance
to burn through the full training set.

Layer 2: I do a max pooling on the output of the convolve, setting up
the pool in such a way that it reduces the output image to an 16x16.

Layer 3: 3 more convolutions, this time doubling the filter count to 32.

Layer 4: Max pool again, outputting an 8x8 image.

Layer 5: 3 more convolutions, doubling the filter count to 64.

Layer 6: Average pool instead of max pool. Max pooling is helpful in
highlighting distinct pixels nestled in a sea of static values. At this
layer I am interested in taking the average, smoothing out the image and
leveraging more information in my final output, a 4x4 image.

Layer 7: I flatten the pool, preparing to generate the logits.

Layer 8: To generate teh fully connected output I perform a linear
operation and define the logits.

\begin{figure}
\centering
\includegraphics{vgg16.png}
\caption{alt text}
\end{figure}

    \hypertarget{train-validate-and-test-the-model}{%
\subsubsection{Train, Validate and Test the
Model}\label{train-validate-and-test-the-model}}

    A validation set can be used to assess how well the model is performing.
A low accuracy on the training and validation sets imply underfitting. A
high accuracy on the training set but low accuracy on the validation set
implies overfitting.

    \begin{Verbatim}[commandchars=\\\{\}]
{\color{incolor}In [{\color{incolor}6}]:} \PY{c+c1}{\PYZsh{}\PYZsh{}\PYZsh{} Train your model here.}
        \PY{c+c1}{\PYZsh{}\PYZsh{}\PYZsh{} Calculate and report the accuracy on the training and validation set.}
        \PY{c+c1}{\PYZsh{}\PYZsh{}\PYZsh{} Once a final model architecture is selected, }
        \PY{c+c1}{\PYZsh{}\PYZsh{}\PYZsh{} the accuracy on the test set should be calculated and reported as well.}
        \PY{c+c1}{\PYZsh{}\PYZsh{}\PYZsh{} Feel free to use as many code cells as needed.}
        \PY{k+kn}{from} \PY{n+nn}{sklearn}\PY{n+nn}{.}\PY{n+nn}{utils} \PY{k}{import} \PY{n}{shuffle}
        \PY{k+kn}{from} \PY{n+nn}{tqdm} \PY{k}{import} \PY{n}{tqdm}
        
        \PY{n}{rate} \PY{o}{=} \PY{l+m+mf}{0.001}
        \PY{n}{EPOCHS} \PY{o}{=} \PY{l+m+mi}{10}
        \PY{n}{BATCH\PYZus{}SIZE} \PY{o}{=} \PY{l+m+mi}{128}
        
        
        \PY{n}{var\PYZus{}rate} \PY{o}{=} \PY{n}{tf}\PY{o}{.}\PY{n}{placeholder}\PY{p}{(}\PY{n}{tf}\PY{o}{.}\PY{n}{float32}\PY{p}{,} \PY{p}{[}\PY{p}{]}\PY{p}{)}
        
        \PY{n}{cross\PYZus{}entropy} \PY{o}{=} \PY{n}{tf}\PY{o}{.}\PY{n}{nn}\PY{o}{.}\PY{n}{sparse\PYZus{}softmax\PYZus{}cross\PYZus{}entropy\PYZus{}with\PYZus{}logits}\PY{p}{(}\PY{n}{labels}\PY{o}{=}\PY{n}{label\PYZus{}data}\PY{p}{,} \PY{n}{logits}\PY{o}{=}\PY{n}{logits}\PY{p}{)}
        \PY{n}{loss\PYZus{}operation} \PY{o}{=} \PY{n}{tf}\PY{o}{.}\PY{n}{reduce\PYZus{}mean}\PY{p}{(}\PY{n}{cross\PYZus{}entropy}\PY{p}{)}
        
        \PY{n}{optimizer} \PY{o}{=} \PY{n}{tf}\PY{o}{.}\PY{n}{train}\PY{o}{.}\PY{n}{AdamOptimizer}\PY{p}{(}\PY{n}{learning\PYZus{}rate} \PY{o}{=} \PY{n}{var\PYZus{}rate}\PY{p}{)}
        \PY{n}{training\PYZus{}operation} \PY{o}{=} \PY{n}{optimizer}\PY{o}{.}\PY{n}{minimize}\PY{p}{(}\PY{n}{loss\PYZus{}operation}\PY{p}{)}
        
        \PY{n}{correct\PYZus{}prediction} \PY{o}{=} \PY{n}{tf}\PY{o}{.}\PY{n}{equal}\PY{p}{(}\PY{n}{tf}\PY{o}{.}\PY{n}{argmax}\PY{p}{(}\PY{n}{logits}\PY{p}{,} \PY{l+m+mi}{1}\PY{p}{)}\PY{p}{,} \PY{n}{tf}\PY{o}{.}\PY{n}{cast}\PY{p}{(}\PY{n}{label\PYZus{}data}\PY{p}{,} \PY{n}{tf}\PY{o}{.}\PY{n}{int64}\PY{p}{)}\PY{p}{)}
        \PY{n}{accuracy\PYZus{}operation} \PY{o}{=} \PY{n}{tf}\PY{o}{.}\PY{n}{reduce\PYZus{}mean}\PY{p}{(}\PY{n}{tf}\PY{o}{.}\PY{n}{cast}\PY{p}{(}\PY{n}{correct\PYZus{}prediction}\PY{p}{,} \PY{n}{tf}\PY{o}{.}\PY{n}{float32}\PY{p}{)}\PY{p}{)}
        \PY{n}{saver} \PY{o}{=} \PY{n}{tf}\PY{o}{.}\PY{n}{train}\PY{o}{.}\PY{n}{Saver}\PY{p}{(}\PY{p}{)}
        
        \PY{k}{def} \PY{n+nf}{evaluate}\PY{p}{(}\PY{n}{X\PYZus{}data}\PY{p}{,} \PY{n}{y\PYZus{}data}\PY{p}{)}\PY{p}{:}
            \PY{n}{num\PYZus{}examples} \PY{o}{=} \PY{n+nb}{len}\PY{p}{(}\PY{n}{X\PYZus{}data}\PY{p}{)}
            \PY{n}{total\PYZus{}accuracy} \PY{o}{=} \PY{l+m+mi}{0}
            \PY{n}{sess} \PY{o}{=} \PY{n}{tf}\PY{o}{.}\PY{n}{get\PYZus{}default\PYZus{}session}\PY{p}{(}\PY{p}{)}
            \PY{k}{for} \PY{n}{offset} \PY{o+ow}{in} \PY{n+nb}{range}\PY{p}{(}\PY{l+m+mi}{0}\PY{p}{,} \PY{n}{num\PYZus{}examples}\PY{p}{,} \PY{n}{BATCH\PYZus{}SIZE}\PY{p}{)}\PY{p}{:}
                \PY{n}{batch\PYZus{}x}\PY{p}{,} \PY{n}{batch\PYZus{}y} \PY{o}{=} \PY{n}{X\PYZus{}data}\PY{p}{[}\PY{n}{offset}\PY{p}{:}\PY{n}{offset}\PY{o}{+}\PY{n}{BATCH\PYZus{}SIZE}\PY{p}{]}\PY{p}{,} \PY{n}{y\PYZus{}data}\PY{p}{[}\PY{n}{offset}\PY{p}{:}\PY{n}{offset}\PY{o}{+}\PY{n}{BATCH\PYZus{}SIZE}\PY{p}{]}
                \PY{n}{accuracy} \PY{o}{=} \PY{n}{sess}\PY{o}{.}\PY{n}{run}\PY{p}{(}\PY{n}{accuracy\PYZus{}operation}\PY{p}{,} \PY{n}{feed\PYZus{}dict}\PY{o}{=}\PY{p}{\PYZob{}}\PY{n}{input\PYZus{}data}\PY{p}{:} \PY{n}{batch\PYZus{}x}\PY{p}{,} \PY{n}{label\PYZus{}data}\PY{p}{:} \PY{n}{batch\PYZus{}y}\PY{p}{\PYZcb{}}\PY{p}{)}
                \PY{n}{total\PYZus{}accuracy} \PY{o}{+}\PY{o}{=} \PY{p}{(}\PY{n}{accuracy} \PY{o}{*} \PY{n+nb}{len}\PY{p}{(}\PY{n}{batch\PYZus{}x}\PY{p}{)}\PY{p}{)}
            \PY{k}{return} \PY{n}{total\PYZus{}accuracy} \PY{o}{/} \PY{n}{num\PYZus{}examples}
        \PY{n}{counter} \PY{o}{=} \PY{l+m+mi}{0}
        \PY{k}{with} \PY{n}{tf}\PY{o}{.}\PY{n}{Session}\PY{p}{(}\PY{p}{)} \PY{k}{as} \PY{n}{sess}\PY{p}{:}
            \PY{n}{sess}\PY{o}{.}\PY{n}{run}\PY{p}{(}\PY{n}{tf}\PY{o}{.}\PY{n}{global\PYZus{}variables\PYZus{}initializer}\PY{p}{(}\PY{p}{)}\PY{p}{)}
            \PY{n}{num\PYZus{}examples} \PY{o}{=} \PY{n+nb}{len}\PY{p}{(}\PY{n}{X\PYZus{}train}\PY{p}{)}
            
            \PY{n+nb}{print}\PY{p}{(}\PY{l+s+s2}{\PYZdq{}}\PY{l+s+s2}{Training...}\PY{l+s+s2}{\PYZdq{}}\PY{p}{)}
            \PY{n+nb}{print}\PY{p}{(}\PY{p}{)}
            \PY{k}{for} \PY{n}{i} \PY{o+ow}{in} \PY{n+nb}{range}\PY{p}{(}\PY{n}{EPOCHS}\PY{p}{)}\PY{p}{:}
                \PY{n}{X\PYZus{}train}\PY{p}{,} \PY{n}{y\PYZus{}train} \PY{o}{=} \PY{n}{shuffle}\PY{p}{(}\PY{n}{X\PYZus{}train}\PY{p}{,} \PY{n}{y\PYZus{}train}\PY{p}{)}
                \PY{k}{for} \PY{n}{offset} \PY{o+ow}{in} \PY{n}{tqdm}\PY{p}{(}\PY{n+nb}{list}\PY{p}{(}\PY{n+nb}{range}\PY{p}{(}\PY{l+m+mi}{0}\PY{p}{,} \PY{n}{num\PYZus{}examples}\PY{p}{,} \PY{n}{BATCH\PYZus{}SIZE}\PY{p}{)}\PY{p}{)}\PY{p}{)}\PY{p}{:}
                    \PY{n}{end} \PY{o}{=} \PY{n}{offset} \PY{o}{+} \PY{n}{BATCH\PYZus{}SIZE}
                    \PY{n}{batch\PYZus{}x}\PY{p}{,} \PY{n}{batch\PYZus{}y} \PY{o}{=} \PY{n}{X\PYZus{}train}\PY{p}{[}\PY{n}{offset}\PY{p}{:}\PY{n}{end}\PY{p}{]}\PY{p}{,} \PY{n}{y\PYZus{}train}\PY{p}{[}\PY{n}{offset}\PY{p}{:}\PY{n}{end}\PY{p}{]}
                    \PY{n}{sess}\PY{o}{.}\PY{n}{run}\PY{p}{(}\PY{n}{training\PYZus{}operation}\PY{p}{,} \PY{n}{feed\PYZus{}dict}\PY{o}{=}\PY{p}{\PYZob{}}\PY{n}{input\PYZus{}data}\PY{p}{:} \PY{n}{batch\PYZus{}x}\PY{p}{,} \PY{n}{label\PYZus{}data}\PY{p}{:} \PY{n}{batch\PYZus{}y}\PY{p}{,} \PY{n}{var\PYZus{}rate}\PY{p}{:} \PY{n}{rate}\PY{p}{\PYZcb{}}\PY{p}{)}
                    \PY{n}{counter} \PY{o}{+}\PY{o}{=} \PY{l+m+mi}{1}
                \PY{n}{validation\PYZus{}accuracy} \PY{o}{=} \PY{n}{evaluate}\PY{p}{(}\PY{n}{X\PYZus{}valid}\PY{p}{,} \PY{n}{y\PYZus{}valid}\PY{p}{)}
                \PY{n+nb}{print}\PY{p}{(}\PY{l+s+s2}{\PYZdq{}}\PY{l+s+s2}{EPOCH }\PY{l+s+si}{\PYZob{}\PYZcb{}}\PY{l+s+s2}{ ...}\PY{l+s+s2}{\PYZdq{}}\PY{o}{.}\PY{n}{format}\PY{p}{(}\PY{n}{i}\PY{o}{+}\PY{l+m+mi}{1}\PY{p}{)}\PY{p}{)}
                \PY{n+nb}{print}\PY{p}{(}\PY{l+s+s2}{\PYZdq{}}\PY{l+s+s2}{Validation Accuracy = }\PY{l+s+si}{\PYZob{}:.3f\PYZcb{}}\PY{l+s+s2}{\PYZdq{}}\PY{o}{.}\PY{n}{format}\PY{p}{(}\PY{n}{validation\PYZus{}accuracy}\PY{p}{)}\PY{p}{)}
                \PY{n+nb}{print}\PY{p}{(}\PY{p}{)}
                
            \PY{n}{saver}\PY{o}{.}\PY{n}{save}\PY{p}{(}\PY{n}{sess}\PY{p}{,} \PY{l+s+s1}{\PYZsq{}}\PY{l+s+s1}{./baseline\PYZhy{}traffic\PYZhy{}sign\PYZhy{}classifer}\PY{l+s+s1}{\PYZsq{}}\PY{p}{)}
            \PY{n+nb}{print}\PY{p}{(}\PY{l+s+s2}{\PYZdq{}}\PY{l+s+s2}{Model saved}\PY{l+s+s2}{\PYZdq{}}\PY{p}{)}
            
        \PY{k}{with} \PY{n}{tf}\PY{o}{.}\PY{n}{Session}\PY{p}{(}\PY{p}{)} \PY{k}{as} \PY{n}{sess}\PY{p}{:}
            \PY{n}{saver}\PY{o}{.}\PY{n}{restore}\PY{p}{(}\PY{n}{sess}\PY{p}{,} \PY{n}{tf}\PY{o}{.}\PY{n}{train}\PY{o}{.}\PY{n}{latest\PYZus{}checkpoint}\PY{p}{(}\PY{l+s+s1}{\PYZsq{}}\PY{l+s+s1}{.}\PY{l+s+s1}{\PYZsq{}}\PY{p}{)}\PY{p}{)}
        
            \PY{n}{test\PYZus{}accuracy} \PY{o}{=} \PY{n}{evaluate}\PY{p}{(}\PY{n}{X\PYZus{}test}\PY{p}{,} \PY{n}{y\PYZus{}test}\PY{p}{)}
            \PY{n+nb}{print}\PY{p}{(}\PY{l+s+s2}{\PYZdq{}}\PY{l+s+s2}{Test Accuracy = }\PY{l+s+si}{\PYZob{}:.3f\PYZcb{}}\PY{l+s+s2}{\PYZdq{}}\PY{o}{.}\PY{n}{format}\PY{p}{(}\PY{n}{test\PYZus{}accuracy}\PY{p}{)}\PY{p}{)}
\end{Verbatim}


    \begin{Verbatim}[commandchars=\\\{\}]
Training{\ldots}


    \end{Verbatim}

    \begin{Verbatim}[commandchars=\\\{\}]
100\%|██████████| 272/272 [00:04<00:00, 56.41it/s]

    \end{Verbatim}

    \begin{Verbatim}[commandchars=\\\{\}]
EPOCH 1 {\ldots}
Validation Accuracy = 0.501


    \end{Verbatim}

    \begin{Verbatim}[commandchars=\\\{\}]
100\%|██████████| 272/272 [00:03<00:00, 68.89it/s]

    \end{Verbatim}

    \begin{Verbatim}[commandchars=\\\{\}]
EPOCH 2 {\ldots}
Validation Accuracy = 0.847


    \end{Verbatim}

    \begin{Verbatim}[commandchars=\\\{\}]
100\%|██████████| 272/272 [00:03<00:00, 68.62it/s]

    \end{Verbatim}

    \begin{Verbatim}[commandchars=\\\{\}]
EPOCH 3 {\ldots}
Validation Accuracy = 0.899


    \end{Verbatim}

    \begin{Verbatim}[commandchars=\\\{\}]
100\%|██████████| 272/272 [00:03<00:00, 69.11it/s]

    \end{Verbatim}

    \begin{Verbatim}[commandchars=\\\{\}]
EPOCH 4 {\ldots}
Validation Accuracy = 0.937


    \end{Verbatim}

    \begin{Verbatim}[commandchars=\\\{\}]
100\%|██████████| 272/272 [00:03<00:00, 69.56it/s]

    \end{Verbatim}

    \begin{Verbatim}[commandchars=\\\{\}]
EPOCH 5 {\ldots}
Validation Accuracy = 0.918


    \end{Verbatim}

    \begin{Verbatim}[commandchars=\\\{\}]
100\%|██████████| 272/272 [00:03<00:00, 68.45it/s]

    \end{Verbatim}

    \begin{Verbatim}[commandchars=\\\{\}]
EPOCH 6 {\ldots}
Validation Accuracy = 0.942


    \end{Verbatim}

    \begin{Verbatim}[commandchars=\\\{\}]
100\%|██████████| 272/272 [00:03<00:00, 69.74it/s]

    \end{Verbatim}

    \begin{Verbatim}[commandchars=\\\{\}]
EPOCH 7 {\ldots}
Validation Accuracy = 0.931


    \end{Verbatim}

    \begin{Verbatim}[commandchars=\\\{\}]
100\%|██████████| 272/272 [00:03<00:00, 68.84it/s]

    \end{Verbatim}

    \begin{Verbatim}[commandchars=\\\{\}]
EPOCH 8 {\ldots}
Validation Accuracy = 0.953


    \end{Verbatim}

    \begin{Verbatim}[commandchars=\\\{\}]
100\%|██████████| 272/272 [00:03<00:00, 69.39it/s]

    \end{Verbatim}

    \begin{Verbatim}[commandchars=\\\{\}]
EPOCH 9 {\ldots}
Validation Accuracy = 0.953


    \end{Verbatim}

    \begin{Verbatim}[commandchars=\\\{\}]
100\%|██████████| 272/272 [00:03<00:00, 68.49it/s]

    \end{Verbatim}

    \begin{Verbatim}[commandchars=\\\{\}]
EPOCH 10 {\ldots}
Validation Accuracy = 0.947

Model saved
INFO:tensorflow:Restoring parameters from ./baseline-traffic-sign-classifer
Test Accuracy = 0.954

    \end{Verbatim}

    \hypertarget{model-training}{%
\paragraph{Model Training}\label{model-training}}

Describe how you trained your model. The discussion can include the type
of optimizer, the batch size, number of epochs and any hyperparameters
such as learning rate.

Optimizer: ADAM

The Momentum Optimzer was tested in training, but was found to take a
little longer to converge than ADAM, but more stable towards the end.
Results ended up not being better than ADAM, though more stable, so I
reverted back to ADAM.

Batch Size: 128

Did not modify.

Epochs: 10

There was a good amount of tweaking of this value. Started with 10,
jumped to 100, began seeing some strange oscillations at various points,
which resulted in modifications to the model architecture. With larger
numbers of filters on the convolutions and more layers (ending with a
2x2 pool instead of a 4x4 pool), the system became unstable in the
mid-teens of epochs. This epoch number was chosen after moving to a
variable learning rate, and the variability of the learning rate is tied
to the epoch number.

Learning Rate: Variable with epoch count

The learning rate starts at 0.001, but the validation accuracy would be
unstable and never fully settle with the ADAM optimizer. To improve the
settling time and stability of the accuracy value, the learning rate was
made variable, slowing down as the epochs increased, allowing the
optimizer to make finer adjustments.

\hypertarget{model-development}{%
\paragraph{Model Development}\label{model-development}}

Describe the approach taken for finding a solution and getting the
validation set accuracy to be at least 0.93. Include in the discussion
the results on the training, validation and test sets and where in the
code these were calculated. Your approach may have been an iterative
process, in which case, outline the steps you took to get to the final
solution and why you chose those steps. Perhaps your solution involved
an already well known implementation or architecture. In this case,
discuss why you think the architecture is suitable for the current
problem. My final model results were:

training set accuracy of ? validation set accuracy of ? test set
accuracy of ? If an iterative approach was chosen:

My approach was a hybrid of the iterative approach and choosing a
well-known architecture. I essentially implemented the VGG16
architecture, and I chose it mostly because it was relatively easy to
implement, and it was recommended to me by a friend.

Unfortunately, the exact architecture of VGG16 was too computationally
expensive for me, and, later, when testing with larger numbers of
filters, it proved to be unnecessary for this data set.

VGG16 is meant for image recognition, with a 7.3\% top-5 score on
ImageNet, making significantly better than LeNet.

Given LeNet was estimated to have a 0.89 validation accuracy and this
network, while not fully VGG16, performs better with 94.7\% validation
accuracy and a 95.4\% test accuracy..

    \begin{center}\rule{0.5\linewidth}{\linethickness}\end{center}

\hypertarget{step-3-test-a-model-on-new-images}{%
\subsection{Step 3: Test a Model on New
Images}\label{step-3-test-a-model-on-new-images}}

To give yourself more insight into how your model is working, download
at least five pictures of German traffic signs from the web and use your
model to predict the traffic sign type.

You may find \texttt{signnames.csv} useful as it contains mappings from
the class id (integer) to the actual sign name.

    \hypertarget{load-and-output-the-images}{%
\subsubsection{Load and Output the
Images}\label{load-and-output-the-images}}

    \begin{Verbatim}[commandchars=\\\{\}]
{\color{incolor}In [{\color{incolor}7}]:} \PY{c+c1}{\PYZsh{}\PYZsh{}\PYZsh{} Load the images and plot them here.}
        \PY{c+c1}{\PYZsh{}\PYZsh{}\PYZsh{} Feel free to use as many code cells as needed.}
        
        \PY{c+c1}{\PYZsh{}reading in an image}
        \PY{k+kn}{import} \PY{n+nn}{glob}
        \PY{k+kn}{import} \PY{n+nn}{matplotlib}\PY{n+nn}{.}\PY{n+nn}{image} \PY{k}{as} \PY{n+nn}{mpimg}
        \PY{k+kn}{import} \PY{n+nn}{cv2}
        
        \PY{n}{fig}\PY{p}{,} \PY{n}{axs} \PY{o}{=} \PY{n}{plt}\PY{o}{.}\PY{n}{subplots}\PY{p}{(}\PY{l+m+mi}{1}\PY{p}{,}\PY{l+m+mi}{5}\PY{p}{,} \PY{n}{figsize}\PY{o}{=}\PY{p}{(}\PY{l+m+mi}{10}\PY{p}{,} \PY{l+m+mi}{5}\PY{p}{)}\PY{p}{)}
        \PY{n}{axs} \PY{o}{=} \PY{n}{axs}\PY{o}{.}\PY{n}{ravel}\PY{p}{(}\PY{p}{)}
        
        \PY{n}{new\PYZus{}train} \PY{o}{=} \PY{p}{[}\PY{p}{]}
        
        \PY{k}{for} \PY{n}{i}\PY{p}{,} \PY{n}{img} \PY{o+ow}{in} \PY{n+nb}{enumerate}\PY{p}{(}\PY{n}{glob}\PY{o}{.}\PY{n}{glob}\PY{p}{(}\PY{l+s+s1}{\PYZsq{}}\PY{l+s+s1}{./New\PYZus{}Sign\PYZus{}Testing/*.jpg}\PY{l+s+s1}{\PYZsq{}}\PY{p}{)}\PY{p}{)}\PY{p}{:}
            \PY{n}{image} \PY{o}{=} \PY{n}{cv2}\PY{o}{.}\PY{n}{imread}\PY{p}{(}\PY{n}{img}\PY{p}{)}
            \PY{n}{image} \PY{o}{=} \PY{n}{cv2}\PY{o}{.}\PY{n}{resize}\PY{p}{(}\PY{n}{image}\PY{p}{,} \PY{p}{(}\PY{l+m+mi}{32}\PY{p}{,} \PY{l+m+mi}{32}\PY{p}{)}\PY{p}{,} \PY{n}{interpolation}\PY{o}{=}\PY{n}{cv2}\PY{o}{.}\PY{n}{INTER\PYZus{}AREA}\PY{p}{)}
            \PY{n}{axs}\PY{p}{[}\PY{n}{i}\PY{p}{]}\PY{o}{.}\PY{n}{axis}\PY{p}{(}\PY{l+s+s1}{\PYZsq{}}\PY{l+s+s1}{off}\PY{l+s+s1}{\PYZsq{}}\PY{p}{)}
            \PY{n}{axs}\PY{p}{[}\PY{n}{i}\PY{p}{]}\PY{o}{.}\PY{n}{imshow}\PY{p}{(}\PY{n}{cv2}\PY{o}{.}\PY{n}{cvtColor}\PY{p}{(}\PY{n}{image}\PY{p}{,} \PY{n}{cv2}\PY{o}{.}\PY{n}{COLOR\PYZus{}BGR2RGB}\PY{p}{)}\PY{p}{)}
            \PY{n}{new\PYZus{}train}\PY{o}{.}\PY{n}{append}\PY{p}{(}\PY{n}{image}\PY{p}{)}
        
        \PY{c+c1}{\PYZsh{}https://github.com/jeremy\PYZhy{}shannon/CarND\PYZhy{}Traffic\PYZhy{}Sign\PYZhy{}Classifier\PYZhy{}Project/blob/master/Traffic\PYZus{}Sign\PYZus{}Classifier.ipynb}
\end{Verbatim}


    \begin{center}
    \adjustimage{max size={0.9\linewidth}{0.9\paperheight}}{output_24_0.png}
    \end{center}
    { \hspace*{\fill} \\}
    
    \hypertarget{new-test-images}{%
\paragraph{New Test Images}\label{new-test-images}}

I wanted to test how robust the neural network was at classifying German
traffic signs so I selected images that had more variance in the
background image, had the signs set at an off-angle, and with varying
levels of lightness. All of these images were scaled and cropped to the
appropriate size, but intentionally chosen to be different from the
training set.

    \hypertarget{predict-the-sign-type-for-each-image}{%
\subsubsection{Predict the Sign Type for Each
Image}\label{predict-the-sign-type-for-each-image}}

    \begin{Verbatim}[commandchars=\\\{\}]
{\color{incolor}In [{\color{incolor}8}]:} \PY{c+c1}{\PYZsh{}\PYZsh{}\PYZsh{} Run the predictions here and use the model to output the prediction for each image.}
        \PY{c+c1}{\PYZsh{}\PYZsh{}\PYZsh{} Make sure to pre\PYZhy{}process the images with the same pre\PYZhy{}processing pipeline used earlier.}
        \PY{c+c1}{\PYZsh{}\PYZsh{}\PYZsh{} Feel free to use as many code cells as needed.}
        
        \PY{n}{new\PYZus{}train} \PY{o}{=} \PY{n}{np}\PY{o}{.}\PY{n}{asarray}\PY{p}{(}\PY{n}{new\PYZus{}train}\PY{p}{)}
        
        \PY{n}{new\PYZus{}train\PYZus{}normalized} \PY{o}{=} \PY{p}{(}\PY{n}{new\PYZus{}train} \PY{o}{\PYZhy{}} \PY{l+m+mf}{127.5}\PY{p}{)}\PY{o}{/}\PY{l+m+mf}{127.5} 
        
        \PY{k}{with} \PY{n}{tf}\PY{o}{.}\PY{n}{Session}\PY{p}{(}\PY{p}{)} \PY{k}{as} \PY{n}{sess}\PY{p}{:}
            \PY{n}{saver}\PY{o}{.}\PY{n}{restore}\PY{p}{(}\PY{n}{sess}\PY{p}{,} \PY{n}{tf}\PY{o}{.}\PY{n}{train}\PY{o}{.}\PY{n}{latest\PYZus{}checkpoint}\PY{p}{(}\PY{l+s+s1}{\PYZsq{}}\PY{l+s+s1}{.}\PY{l+s+s1}{\PYZsq{}}\PY{p}{)}\PY{p}{)}
            \PY{n}{num\PYZus{}examples} \PY{o}{=} \PY{n+nb}{len}\PY{p}{(}\PY{n}{new\PYZus{}train\PYZus{}normalized}\PY{p}{)}
            \PY{n}{pred\PYZus{}logits} \PY{o}{=} \PY{n}{sess}\PY{o}{.}\PY{n}{run}\PY{p}{(}\PY{n}{logits}\PY{p}{,} \PY{n}{feed\PYZus{}dict}\PY{o}{=}\PY{p}{\PYZob{}}\PY{n}{input\PYZus{}data}\PY{p}{:} \PY{n}{new\PYZus{}train\PYZus{}normalized}\PY{p}{\PYZcb{}}\PY{p}{)}
            \PY{c+c1}{\PYZsh{}Print the predicted class for each of the images}
            \PY{n+nb}{print}\PY{p}{(}\PY{n}{np}\PY{o}{.}\PY{n}{argmax}\PY{p}{(}\PY{n}{pred\PYZus{}logits}\PY{p}{,} \PY{l+m+mi}{1}\PY{p}{)}\PY{p}{)}
\end{Verbatim}


    \begin{Verbatim}[commandchars=\\\{\}]
INFO:tensorflow:Restoring parameters from ./baseline-traffic-sign-classifer
[38  7 39  4 13]

    \end{Verbatim}

    \hypertarget{analyze-performance}{%
\subsubsection{Analyze Performance}\label{analyze-performance}}

    \begin{Verbatim}[commandchars=\\\{\}]
{\color{incolor}In [{\color{incolor}9}]:} \PY{c+c1}{\PYZsh{}\PYZsh{}\PYZsh{} Calculate the accuracy for these 5 new images. }
        \PY{c+c1}{\PYZsh{}\PYZsh{}\PYZsh{} For example, if the model predicted 1 out of 5 signs correctly, it\PYZsq{}s 20\PYZpc{} accurate on these new images.}
        
        \PY{k+kn}{import} \PY{n+nn}{tensorflow} \PY{k}{as} \PY{n+nn}{tf}
        
        \PY{n}{read\PYZus{}labels} \PY{o}{=} \PY{p}{[}\PY{l+m+mi}{26}\PY{p}{,}\PY{l+m+mi}{18}\PY{p}{,}\PY{l+m+mi}{13}\PY{p}{,}\PY{l+m+mi}{5}\PY{p}{,}\PY{l+m+mi}{14}\PY{p}{]}
        
        \PY{k}{with} \PY{n}{tf}\PY{o}{.}\PY{n}{Session}\PY{p}{(}\PY{p}{)} \PY{k}{as} \PY{n}{sess}\PY{p}{:}
            \PY{n}{sess}\PY{o}{.}\PY{n}{run}\PY{p}{(}\PY{n}{tf}\PY{o}{.}\PY{n}{global\PYZus{}variables\PYZus{}initializer}\PY{p}{(}\PY{p}{)}\PY{p}{)}
            \PY{n}{new\PYZus{}saver} \PY{o}{=} \PY{n}{saver}\PY{o}{.}\PY{n}{restore}\PY{p}{(}\PY{n}{sess}\PY{p}{,} \PY{n}{tf}\PY{o}{.}\PY{n}{train}\PY{o}{.}\PY{n}{latest\PYZus{}checkpoint}\PY{p}{(}\PY{l+s+s1}{\PYZsq{}}\PY{l+s+s1}{.}\PY{l+s+s1}{\PYZsq{}}\PY{p}{)}\PY{p}{)}
            \PY{n}{saver}\PY{o}{.}\PY{n}{restore}\PY{p}{(}\PY{n}{sess}\PY{p}{,} \PY{l+s+s1}{\PYZsq{}}\PY{l+s+s1}{./baseline\PYZhy{}traffic\PYZhy{}sign\PYZhy{}classifer}\PY{l+s+s1}{\PYZsq{}}\PY{p}{)}
            \PY{n}{accuracy} \PY{o}{=} \PY{n}{evaluate}\PY{p}{(}\PY{n}{new\PYZus{}train\PYZus{}normalized}\PY{p}{,} \PY{n}{read\PYZus{}labels}\PY{p}{)}
            \PY{n+nb}{print}\PY{p}{(}\PY{l+s+s2}{\PYZdq{}}\PY{l+s+s2}{Test Set Accuracy = }\PY{l+s+si}{\PYZob{}:.3f\PYZcb{}}\PY{l+s+s2}{\PYZdq{}}\PY{o}{.}\PY{n}{format}\PY{p}{(}\PY{n}{accuracy}\PY{p}{)}\PY{p}{)}
\end{Verbatim}


    \begin{Verbatim}[commandchars=\\\{\}]
INFO:tensorflow:Restoring parameters from ./baseline-traffic-sign-classifer
INFO:tensorflow:Restoring parameters from ./baseline-traffic-sign-classifer
Test Set Accuracy = 0.000

    \end{Verbatim}

    \hypertarget{performance-on-new-set}{%
\paragraph{Performance on New Set}\label{performance-on-new-set}}

It appears that the new set of images were too different from the
training set. None of the images were correctly classified. This is
somewhat understandable because the training set is much more curated
and controlled than this new set, but this also gives insight into
potential augmentations I might want to do to the training set to
imrpove robustness. Some options include distorting the images to adjust
for different camera angles, increasing lightness of the image, and
potentially using HSL instead of an RGB colorspace.

    \hypertarget{output-top-5-softmax-probabilities-for-each-image-found-on-the-web}{%
\subsubsection{Output Top 5 Softmax Probabilities For Each Image Found
on the
Web}\label{output-top-5-softmax-probabilities-for-each-image-found-on-the-web}}

    For each of the new images, print out the model's softmax probabilities
to show the \textbf{certainty} of the model's predictions (limit the
output to the top 5 probabilities for each image).
\href{https://www.tensorflow.org/versions/r0.12/api_docs/python/nn.html\#top_k}{\texttt{tf.nn.top\_k}}
could prove helpful here.

The example below demonstrates how tf.nn.top\_k can be used to find the
top k predictions for each image.

\texttt{tf.nn.top\_k} will return the values and indices (class ids) of
the top k predictions. So if k=3, for each sign, it'll return the 3
largest probabilities (out of a possible 43) and the correspoding class
ids.

Take this numpy array as an example. The values in the array represent
predictions. The array contains softmax probabilities for five candidate
images with six possible classes. \texttt{tf.nn.top\_k} is used to
choose the three classes with the highest probability:

\begin{verbatim}
# (5, 6) array
a = np.array([[ 0.24879643,  0.07032244,  0.12641572,  0.34763842,  0.07893497,
         0.12789202],
       [ 0.28086119,  0.27569815,  0.08594638,  0.0178669 ,  0.18063401,
         0.15899337],
       [ 0.26076848,  0.23664738,  0.08020603,  0.07001922,  0.1134371 ,
         0.23892179],
       [ 0.11943333,  0.29198961,  0.02605103,  0.26234032,  0.1351348 ,
         0.16505091],
       [ 0.09561176,  0.34396535,  0.0643941 ,  0.16240774,  0.24206137,
         0.09155967]])
\end{verbatim}

Running it through \texttt{sess.run(tf.nn.top\_k(tf.constant(a),\ k=3))}
produces:

\begin{verbatim}
TopKV2(values=array([[ 0.34763842,  0.24879643,  0.12789202],
       [ 0.28086119,  0.27569815,  0.18063401],
       [ 0.26076848,  0.23892179,  0.23664738],
       [ 0.29198961,  0.26234032,  0.16505091],
       [ 0.34396535,  0.24206137,  0.16240774]]), indices=array([[3, 0, 5],
       [0, 1, 4],
       [0, 5, 1],
       [1, 3, 5],
       [1, 4, 3]], dtype=int32))
\end{verbatim}

Looking just at the first row we get
\texttt{{[}\ 0.34763842,\ \ 0.24879643,\ \ 0.12789202{]}}, you can
confirm these are the 3 largest probabilities in \texttt{a}. You'll also
notice \texttt{{[}3,\ 0,\ 5{]}} are the corresponding indices.

    \begin{Verbatim}[commandchars=\\\{\}]
{\color{incolor}In [{\color{incolor}10}]:} \PY{c+c1}{\PYZsh{}\PYZsh{}\PYZsh{} Print out the top five softmax probabilities for the predictions on the German traffic sign images found on the web. }
         \PY{c+c1}{\PYZsh{}\PYZsh{}\PYZsh{} Feel free to use as many code cells as needed.}
         
         \PY{n}{softmax} \PY{o}{=} \PY{n}{tf}\PY{o}{.}\PY{n}{nn}\PY{o}{.}\PY{n}{softmax}\PY{p}{(}\PY{n}{logits}\PY{p}{)}
         \PY{n}{top\PYZus{}5} \PY{o}{=} \PY{n}{tf}\PY{o}{.}\PY{n}{nn}\PY{o}{.}\PY{n}{top\PYZus{}k}\PY{p}{(}\PY{n}{softmax}\PY{p}{,} \PY{n}{k}\PY{o}{=}\PY{l+m+mi}{5}\PY{p}{)}
         
         \PY{k}{with} \PY{n}{tf}\PY{o}{.}\PY{n}{Session}\PY{p}{(}\PY{p}{)} \PY{k}{as} \PY{n}{sess}\PY{p}{:}
             \PY{n}{sess}\PY{o}{.}\PY{n}{run}\PY{p}{(}\PY{n}{tf}\PY{o}{.}\PY{n}{global\PYZus{}variables\PYZus{}initializer}\PY{p}{(}\PY{p}{)}\PY{p}{)}
             \PY{n}{new\PYZus{}saver} \PY{o}{=} \PY{n}{saver}\PY{o}{.}\PY{n}{restore}\PY{p}{(}\PY{n}{sess}\PY{p}{,} \PY{n}{tf}\PY{o}{.}\PY{n}{train}\PY{o}{.}\PY{n}{latest\PYZus{}checkpoint}\PY{p}{(}\PY{l+s+s1}{\PYZsq{}}\PY{l+s+s1}{.}\PY{l+s+s1}{\PYZsq{}}\PY{p}{)}\PY{p}{)}
             \PY{n}{saver}\PY{o}{.}\PY{n}{restore}\PY{p}{(}\PY{n}{sess}\PY{p}{,} \PY{l+s+s1}{\PYZsq{}}\PY{l+s+s1}{./baseline\PYZhy{}traffic\PYZhy{}sign\PYZhy{}classifer}\PY{l+s+s1}{\PYZsq{}}\PY{p}{)}
             \PY{n}{softmax\PYZus{}calc} \PY{o}{=} \PY{n}{sess}\PY{o}{.}\PY{n}{run}\PY{p}{(}\PY{n}{softmax}\PY{p}{,} \PY{n}{feed\PYZus{}dict}\PY{o}{=}\PY{p}{\PYZob{}}\PY{n}{input\PYZus{}data}\PY{p}{:} \PY{n}{new\PYZus{}train\PYZus{}normalized}\PY{p}{\PYZcb{}}\PY{p}{)}
             \PY{n}{top\PYZus{}5\PYZus{}out\PYZus{}i}\PY{p}{,} \PY{n}{top\PYZus{}5\PYZus{}prob} \PY{o}{=} \PY{n}{sess}\PY{o}{.}\PY{n}{run}\PY{p}{(}\PY{n}{top\PYZus{}5}\PY{p}{,} \PY{n}{feed\PYZus{}dict}\PY{o}{=}\PY{p}{\PYZob{}}\PY{n}{input\PYZus{}data}\PY{p}{:} \PY{n}{new\PYZus{}train\PYZus{}normalized}\PY{p}{\PYZcb{}}\PY{p}{)}
             \PY{c+c1}{\PYZsh{}print(softmax\PYZus{}calc)}
             \PY{c+c1}{\PYZsh{}print(top\PYZus{}5\PYZus{}out)}
             \PY{n}{merged} \PY{o}{=} \PY{n}{np}\PY{o}{.}\PY{n}{stack}\PY{p}{(}\PY{p}{[}\PY{n}{top\PYZus{}5\PYZus{}out\PYZus{}i}\PY{p}{,} \PY{n}{top\PYZus{}5\PYZus{}prob}\PY{p}{]}\PY{p}{,} \PY{l+m+mi}{2}\PY{p}{)}
             \PY{k}{for} \PY{n}{index1}\PY{p}{,}\PY{n}{output} \PY{o+ow}{in} \PY{n+nb}{enumerate}\PY{p}{(}\PY{n}{merged}\PY{p}{)}\PY{p}{:}
                 \PY{n+nb}{print}\PY{p}{(}\PY{l+s+s2}{\PYZdq{}}\PY{l+s+s2}{Image:}\PY{l+s+s2}{\PYZdq{}}\PY{p}{,} \PY{n}{index1}\PY{o}{+}\PY{l+m+mi}{1}\PY{p}{)}
                 \PY{k}{for} \PY{n}{index2}\PY{p}{,} \PY{p}{(}\PY{n}{prob}\PY{p}{,}\PY{n}{predict}\PY{p}{)} \PY{o+ow}{in} \PY{n+nb}{enumerate}\PY{p}{(}\PY{n}{output}\PY{p}{)}\PY{p}{:}
                     \PY{n+nb}{print}\PY{p}{(}\PY{l+s+s2}{\PYZdq{}}\PY{l+s+s2}{Rank: }\PY{l+s+si}{\PYZpc{}d}\PY{l+s+s2}{ Predicted Class: }\PY{l+s+si}{\PYZpc{}d}\PY{l+s+s2}{ Probability: }\PY{l+s+si}{\PYZpc{}2.2f}\PY{l+s+s2}{\PYZdq{}} \PY{o}{\PYZpc{}} \PY{p}{(}\PY{n}{index2}\PY{o}{+}\PY{l+m+mi}{1}\PY{p}{,} \PY{n}{predict}\PY{p}{,} \PY{n}{prob}\PY{o}{*}\PY{l+m+mi}{100}\PY{p}{)}\PY{p}{)}
                 \PY{n+nb}{print}\PY{p}{(}\PY{p}{)}
\end{Verbatim}


    \begin{Verbatim}[commandchars=\\\{\}]
INFO:tensorflow:Restoring parameters from ./baseline-traffic-sign-classifer
INFO:tensorflow:Restoring parameters from ./baseline-traffic-sign-classifer
Image: 1
Rank: 1 Predicted Class: 38 Probability: 95.81
Rank: 2 Predicted Class: 11 Probability: 3.69
Rank: 3 Predicted Class: 25 Probability: 0.24
Rank: 4 Predicted Class: 1 Probability: 0.13
Rank: 5 Predicted Class: 28 Probability: 0.10

Image: 2
Rank: 1 Predicted Class: 7 Probability: 47.91
Rank: 2 Predicted Class: 5 Probability: 24.82
Rank: 3 Predicted Class: 34 Probability: 23.62
Rank: 4 Predicted Class: 10 Probability: 2.87
Rank: 5 Predicted Class: 8 Probability: 0.36

Image: 3
Rank: 1 Predicted Class: 39 Probability: 99.30
Rank: 2 Predicted Class: 35 Probability: 0.70
Rank: 3 Predicted Class: 11 Probability: 0.00
Rank: 4 Predicted Class: 32 Probability: 0.00
Rank: 5 Predicted Class: 38 Probability: 0.00

Image: 4
Rank: 1 Predicted Class: 4 Probability: 61.90
Rank: 2 Predicted Class: 8 Probability: 24.54
Rank: 3 Predicted Class: 13 Probability: 8.39
Rank: 4 Predicted Class: 38 Probability: 4.81
Rank: 5 Predicted Class: 34 Probability: 0.30

Image: 5
Rank: 1 Predicted Class: 13 Probability: 99.96
Rank: 2 Predicted Class: 1 Probability: 0.03
Rank: 3 Predicted Class: 12 Probability: 0.01
Rank: 4 Predicted Class: 32 Probability: 0.00
Rank: 5 Predicted Class: 36 Probability: 0.00


    \end{Verbatim}

    \hypertarget{prediction-probabilities}{%
\paragraph{Prediction Probabilities}\label{prediction-probabilities}}

Looking at the probabilities is quite insightful. First, none of the
images contain an accurate label even in their top 5 predictions.
Additionally, the classifier is incredibly confident in these
predictions, with three of the five with 90\%+ probabilities. Looking at
the Taking a higher level view, however, some trends to appear.

Image 1 is interesting in that 25 is the third most likely lablel,
though the confidence in that prediction is quite low, and 25 shares
many similarities with the actual label. It is likely that the blue of
the background is dominating the classification.

Image 2 shows that the classifier is likely focusing heavily on
outlines, as it has ignore the distinct difference in color
concentration, though the hues are similar, and labeled this as a speed
limit sign with low confidence.

Image 3 this classification likely is influenced more by the dark
colored outline of the house in the background than the sign itself,
especially given the incredibly high confidence of the prediction.

Image 4, the speed sign, was close; the classifier labled it as a 60
kmph sign instead of a 70 kmph.

Image 5 was classified as 14 instead of 13, and while the colors are
quite different, further supporting that the classifier has more heavily
weighted edges and overall shape relative to colors, and the distorted
triangular shape is being interpreted as part of a full diamond.

    \hypertarget{project-writeup}{%
\subsubsection{Project Writeup}\label{project-writeup}}

Once you have completed the code implementation, document your results
in a project writeup using this
\href{https://github.com/udacity/CarND-Traffic-Sign-Classifier-Project/blob/master/writeup_template.md}{template}
as a guide. The writeup can be in a markdown or pdf file.

    \begin{quote}
\textbf{Note}: Once you have completed all of the code implementations
and successfully answered each question above, you may finalize your
work by exporting the iPython Notebook as an HTML document. You can do
this by using the menu above and navigating to \n``,''\textbf{File
-\textgreater{} Download as -\textgreater{} HTML (.html)}. Include the
finished document along with this notebook as your submission.
\end{quote}

    \begin{center}\rule{0.5\linewidth}{\linethickness}\end{center}

\hypertarget{step-4-optional-visualize-the-neural-networks-state-with-test-images}{%
\subsection{Step 4 (Optional): Visualize the Neural Network's State with
Test
Images}\label{step-4-optional-visualize-the-neural-networks-state-with-test-images}}

This Section is not required to complete but acts as an additional
excersise for understaning the output of a neural network's weights.
While neural networks can be a great learning device they are often
referred to as a black box. We can understand what the weights of a
neural network look like better by plotting their feature maps. After
successfully training your neural network you can see what it's feature
maps look like by plotting the output of the network's weight layers in
response to a test stimuli image. From these plotted feature maps, it's
possible to see what characteristics of an image the network finds
interesting. For a sign, maybe the inner network feature maps react with
high activation to the sign's boundary outline or to the contrast in the
sign's painted symbol.

Provided for you below is the function code that allows you to get the
visualization output of any tensorflow weight layer you want. The inputs
to the function should be a stimuli image, one used during training or a
new one you provided, and then the tensorflow variable name that
represents the layer's state during the training process, for instance
if you wanted to see what the
\href{https://classroom.udacity.com/nanodegrees/nd013/parts/fbf77062-5703-404e-b60c-95b78b2f3f9e/modules/6df7ae49-c61c-4bb2-a23e-6527e69209ec/lessons/601ae704-1035-4287-8b11-e2c2716217ad/concepts/d4aca031-508f-4e0b-b493-e7b706120f81}{LeNet
lab's} feature maps looked like for it's second convolutional layer you
could enter conv2 as the tf\_activation variable.

For an example of what feature map outputs look like, check out NVIDIA's
results in their paper
\href{https://devblogs.nvidia.com/parallelforall/deep-learning-self-driving-cars/}{End-to-End
Deep Learning for Self-Driving Cars} in the section Visualization of
internal CNN State. NVIDIA was able to show that their network's inner
weights had high activations to road boundary lines by comparing feature
maps from an image with a clear path to one without. Try experimenting
with a similar test to show that your trained network's weights are
looking for interesting features, whether it's looking at differences in
feature maps from images with or without a sign, or even what feature
maps look like in a trained network vs a completely untrained one on the
same sign image.

Your output should look something like this (above)

    \begin{Verbatim}[commandchars=\\\{\}]
{\color{incolor}In [{\color{incolor}11}]:} \PY{c+c1}{\PYZsh{}\PYZsh{}\PYZsh{} Visualize your network\PYZsq{}s feature maps here.}
         \PY{c+c1}{\PYZsh{}\PYZsh{}\PYZsh{} Feel free to use as many code cells as needed.}
         
         \PY{c+c1}{\PYZsh{} image\PYZus{}input: the test image being fed into the network to produce the feature maps}
         \PY{c+c1}{\PYZsh{} tf\PYZus{}activation: should be a tf variable name used during your training procedure that represents the calculated state of a specific weight layer}
         \PY{c+c1}{\PYZsh{} activation\PYZus{}min/max: can be used to view the activation contrast in more detail, by default matplot sets min and max to the actual min and max values of the output}
         \PY{c+c1}{\PYZsh{} plt\PYZus{}num: used to plot out multiple different weight feature map sets on the same block, just extend the plt number for each new feature map entry}
         
         \PY{k}{def} \PY{n+nf}{outputFeatureMap}\PY{p}{(}\PY{n}{image\PYZus{}input}\PY{p}{,} \PY{n}{tf\PYZus{}activation}\PY{p}{,} \PY{n}{activation\PYZus{}min}\PY{o}{=}\PY{o}{\PYZhy{}}\PY{l+m+mi}{1}\PY{p}{,} \PY{n}{activation\PYZus{}max}\PY{o}{=}\PY{o}{\PYZhy{}}\PY{l+m+mi}{1} \PY{p}{,}\PY{n}{plt\PYZus{}num}\PY{o}{=}\PY{l+m+mi}{1}\PY{p}{)}\PY{p}{:}
             \PY{c+c1}{\PYZsh{} Here make sure to preprocess your image\PYZus{}input in a way your network expects}
             \PY{c+c1}{\PYZsh{} with size, normalization, ect if needed}
             \PY{c+c1}{\PYZsh{} image\PYZus{}input =}
             \PY{c+c1}{\PYZsh{} Note: x should be the same name as your network\PYZsq{}s tensorflow data placeholder variable}
             \PY{c+c1}{\PYZsh{} If you get an error tf\PYZus{}activation is not defined it may be having trouble accessing the variable from inside a function}
             \PY{n}{activation} \PY{o}{=} \PY{n}{tf\PYZus{}activation}\PY{o}{.}\PY{n}{eval}\PY{p}{(}\PY{n}{session}\PY{o}{=}\PY{n}{sess}\PY{p}{,}\PY{n}{feed\PYZus{}dict}\PY{o}{=}\PY{p}{\PYZob{}}\PY{n}{x} \PY{p}{:} \PY{n}{image\PYZus{}input}\PY{p}{\PYZcb{}}\PY{p}{)}
             \PY{n}{featuremaps} \PY{o}{=} \PY{n}{activation}\PY{o}{.}\PY{n}{shape}\PY{p}{[}\PY{l+m+mi}{3}\PY{p}{]}
             \PY{n}{plt}\PY{o}{.}\PY{n}{figure}\PY{p}{(}\PY{n}{plt\PYZus{}num}\PY{p}{,} \PY{n}{figsize}\PY{o}{=}\PY{p}{(}\PY{l+m+mi}{15}\PY{p}{,}\PY{l+m+mi}{15}\PY{p}{)}\PY{p}{)}
             \PY{k}{for} \PY{n}{featuremap} \PY{o+ow}{in} \PY{n+nb}{range}\PY{p}{(}\PY{n}{featuremaps}\PY{p}{)}\PY{p}{:}
                 \PY{n}{plt}\PY{o}{.}\PY{n}{subplot}\PY{p}{(}\PY{l+m+mi}{6}\PY{p}{,}\PY{l+m+mi}{8}\PY{p}{,} \PY{n}{featuremap}\PY{o}{+}\PY{l+m+mi}{1}\PY{p}{)} \PY{c+c1}{\PYZsh{} sets the number of feature maps to show on each row and column}
                 \PY{n}{plt}\PY{o}{.}\PY{n}{title}\PY{p}{(}\PY{l+s+s1}{\PYZsq{}}\PY{l+s+s1}{FeatureMap }\PY{l+s+s1}{\PYZsq{}} \PY{o}{+} \PY{n+nb}{str}\PY{p}{(}\PY{n}{featuremap}\PY{p}{)}\PY{p}{)} \PY{c+c1}{\PYZsh{} displays the feature map number}
                 \PY{k}{if} \PY{n}{activation\PYZus{}min} \PY{o}{!=} \PY{o}{\PYZhy{}}\PY{l+m+mi}{1} \PY{o}{\PYZam{}} \PY{n}{activation\PYZus{}max} \PY{o}{!=} \PY{o}{\PYZhy{}}\PY{l+m+mi}{1}\PY{p}{:}
                     \PY{n}{plt}\PY{o}{.}\PY{n}{imshow}\PY{p}{(}\PY{n}{activation}\PY{p}{[}\PY{l+m+mi}{0}\PY{p}{,}\PY{p}{:}\PY{p}{,}\PY{p}{:}\PY{p}{,} \PY{n}{featuremap}\PY{p}{]}\PY{p}{,} \PY{n}{interpolation}\PY{o}{=}\PY{l+s+s2}{\PYZdq{}}\PY{l+s+s2}{nearest}\PY{l+s+s2}{\PYZdq{}}\PY{p}{,} \PY{n}{vmin} \PY{o}{=}\PY{n}{activation\PYZus{}min}\PY{p}{,} \PY{n}{vmax}\PY{o}{=}\PY{n}{activation\PYZus{}max}\PY{p}{,} \PY{n}{cmap}\PY{o}{=}\PY{l+s+s2}{\PYZdq{}}\PY{l+s+s2}{gray}\PY{l+s+s2}{\PYZdq{}}\PY{p}{)}
                 \PY{k}{elif} \PY{n}{activation\PYZus{}max} \PY{o}{!=} \PY{o}{\PYZhy{}}\PY{l+m+mi}{1}\PY{p}{:}
                     \PY{n}{plt}\PY{o}{.}\PY{n}{imshow}\PY{p}{(}\PY{n}{activation}\PY{p}{[}\PY{l+m+mi}{0}\PY{p}{,}\PY{p}{:}\PY{p}{,}\PY{p}{:}\PY{p}{,} \PY{n}{featuremap}\PY{p}{]}\PY{p}{,} \PY{n}{interpolation}\PY{o}{=}\PY{l+s+s2}{\PYZdq{}}\PY{l+s+s2}{nearest}\PY{l+s+s2}{\PYZdq{}}\PY{p}{,} \PY{n}{vmax}\PY{o}{=}\PY{n}{activation\PYZus{}max}\PY{p}{,} \PY{n}{cmap}\PY{o}{=}\PY{l+s+s2}{\PYZdq{}}\PY{l+s+s2}{gray}\PY{l+s+s2}{\PYZdq{}}\PY{p}{)}
                 \PY{k}{elif} \PY{n}{activation\PYZus{}min} \PY{o}{!=}\PY{o}{\PYZhy{}}\PY{l+m+mi}{1}\PY{p}{:}
                     \PY{n}{plt}\PY{o}{.}\PY{n}{imshow}\PY{p}{(}\PY{n}{activation}\PY{p}{[}\PY{l+m+mi}{0}\PY{p}{,}\PY{p}{:}\PY{p}{,}\PY{p}{:}\PY{p}{,} \PY{n}{featuremap}\PY{p}{]}\PY{p}{,} \PY{n}{interpolation}\PY{o}{=}\PY{l+s+s2}{\PYZdq{}}\PY{l+s+s2}{nearest}\PY{l+s+s2}{\PYZdq{}}\PY{p}{,} \PY{n}{vmin}\PY{o}{=}\PY{n}{activation\PYZus{}min}\PY{p}{,} \PY{n}{cmap}\PY{o}{=}\PY{l+s+s2}{\PYZdq{}}\PY{l+s+s2}{gray}\PY{l+s+s2}{\PYZdq{}}\PY{p}{)}
                 \PY{k}{else}\PY{p}{:}
                     \PY{n}{plt}\PY{o}{.}\PY{n}{imshow}\PY{p}{(}\PY{n}{activation}\PY{p}{[}\PY{l+m+mi}{0}\PY{p}{,}\PY{p}{:}\PY{p}{,}\PY{p}{:}\PY{p}{,} \PY{n}{featuremap}\PY{p}{]}\PY{p}{,} \PY{n}{interpolation}\PY{o}{=}\PY{l+s+s2}{\PYZdq{}}\PY{l+s+s2}{nearest}\PY{l+s+s2}{\PYZdq{}}\PY{p}{,} \PY{n}{cmap}\PY{o}{=}\PY{l+s+s2}{\PYZdq{}}\PY{l+s+s2}{gray}\PY{l+s+s2}{\PYZdq{}}\PY{p}{)}
\end{Verbatim}



    % Add a bibliography block to the postdoc
    
    
    
    \end{document}
